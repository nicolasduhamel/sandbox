\documentclass[a4paper,12pt]{article}

\usepackage[utf8]{inputenc}
\usepackage[T1]{fontenc}
\usepackage{lmodern}
\usepackage[frenchb]{babel}
\usepackage{amssymb}
\usepackage{amsmath}
\usepackage{amsthm}
\usepackage[french,noend,boxruled]{algorithm2e}
\usepackage{hyperref}
\usepackage{xcolor}\hypersetup{linkbordercolor=red}
%\usepackage{graphics}
%\usepackage{enumerate}
\usepackage{epigraph}
% Citation avec nom d'auteur
\usepackage[square]{natbib}
% Réduire les marges
%\usepackage[a4paper]{geometry}
%\geometry{hmargin=3cm,vmargin=4cm}

% Répare le problème des boites trop large dans algorithm
\makeatletter
\renewcommand{\algocf@caption@boxruled}{%
  \hrule
  \hbox to \hsize{%
    \vrule\hskip-0.4pt
    \vbox{   
       \vskip\interspacetitleboxruled%
       \unhbox\algocf@capbox\hfill
       \vskip\interspacetitleboxruled
       }%
     \hskip-0.4pt\vrule%
   }\nointerlineskip%
}%
\makeatother

% Theorem Styles
\newtheorem{theoreme}{Theorème}[section]
\newtheorem{lemme}[theoreme]{Lemme}
\newtheorem{proposition}[theoreme]{Proposition}
\newtheorem{corollaire}[theoreme]{Corollaire}
% Definition Styles
\theoremstyle{definition}
\newtheorem{definition}{Definition}[section]
\newtheorem{exemple}{Exemple}[section]
\theoremstyle{remark}
\newtheorem{remarque}{Remarque}
% Equation numerotation
\numberwithin{equation}{section}

\begin{document}

\title{Prélude de genre 2 \\ \large{Vers un analogue de genre 2 des courbes de Montgomery}}
\author{N. Du Hamel\footnote{\href{mailto:nicolas.duhamel@inria.fr}{nicolas.duhamel@inria.fr}} \\
\emph{sous la direction de \href{http://www.lix.polytechnique.fr/~smith/}{Ben Smith}}}
\maketitle

%\epigraph{If people do not believe that mathematics is simple, it is only because they do not realize how complicated life is.}{\textit{John von Neumann}}

%\epigraph{It may well be doubted whether human ingenuity can construct an enigma of the kind which human ingenuity may not, by proper application, resolve.}{\textit{Edgar Allan Poe}}

\epigraph{La chance ne sourit qu'aux esprits bien préparés.}{\textit{Louis Pasteur}}

%Le présent document constitue le mémoire d'un stage de M2 intitulé \emph{Cryptographic Kummer Surfaces} qui vise à étudier l'arithmétique rapide sur les surfaces de Kummer.
Les courbes de Montgomery ont été introduites en 1987 par Peter Montgomery pour avoir une arithmétique rapide sur certaines courbes elliptiques, sa motivation principale était l'amélioration de la méthode de factorisation à base de courbe elliptique \citep{montgomery}. Les courbes de Montgomery sont les plus utilisé en cryptographie, ceci venant du fait que leur arithmétique est très efficace.

Dans ce mémoire, on s'intéresse à l'arithmétique rapide des courbes de genre 2 et à leurs applications à la cryptographie. On commence par présenter le contexte cryptographique dans lequel on se place et l'on introduit les courbes de genre 2. On présente ensuite les théories de \citet{cassels-Flynn} et \citet{gaudry} sur l'arithmétique des courbes de genre 2; la première ayant pour objectif d'être la plus générale possible et le seconde d'avoir une arithmétique qui puisse être compétitive face aux courbes elliptiques (genre 1).
On enchaine par une traduction partielle entre ces deux théories, ce qui nous permettra d'avoir un point de vue légèrement différent sur l'arithmétique rapide de genre 2; et pour finir, on spécialise notre étude aux courbes de genre 2 à multiplication réelle par $\sqrt{2}$ pour avoir une arithmétique légèrement plus rapide.
%\tableofcontents

\section{Cryptographie}

La cryptographie désigne un ensemble de méthodes permettant, entre autres, d'assurer l'authentification et la confidentialité des communications. On retrouve des utilisations de la cryptographie dans la vie de tous les jours; par exemple, par l'intermédiaire des cartes bancaires ou des sites web sécurisés (le fameux protocole HTTPS).

Notre vie privée est mise sous surveillance par certaines entités (gouvernementales ou non) qui possèdent des capacités de collectes et de traitements des données inimaginables\footnote{Assume your adversary is capable of one trillion guesses per second. -- Edward Snowden}.
La cryptographie est un moyen de se protéger contre ces pratiques qui viennent à se généraliser.

\subsection{Diffie-Hellman}
\cite{diffie-hellman} ont introduit le concept de \emph{cryptographie à clé publique}, que l'on appelle aussi \emph{cryptographie asymétrique}. Contrairement à la cryptographie symétrique, où les deux parties communiquent en ayant au préalable convenu d'une clé secrète, chaque entité dispose d'une clé publique et d'une clé privée. La clé publique est diffusée sur le réseau, l'expéditeur l'utilise pour chiffrer son message; tandis que le destinataire utilise sa clé privée pour déchiffrer le message. D'autre part, l'expéditeur a la possibilité de signer le message avec sa clé privée; le destinataire utilise alors la clé publique de l'expéditeur pour vérifier l'authenticité du message.

Les implémentations de protocole à clé publique sont plus lentes que leurs homologues symétriques; c'est pourquoi en pratique, la cryptographie à clé publique est souvent utilisée dans le seul but de partager une clé secrète qui pourra ensuite servir de base pour des protocoles symétriques.

On se contentera de décrire le \emph{protocole d'échange de clés Diffie-Hellman}; le but étant de donner une motivation au problème logarithme discret et non pas de faire un état de l'art de la cryptographie à clé publique.

La méthode d'échange de clés de Diffie-Hellman permet à deux parties, qui n'ont aucune connaissance commune à priori, de partager une clé secrète à travers un canal non sécurisé. Pour être plus parlant, ont introduit deux protagonistes Alice et Bob qui désirent partager une clé secrète sans que Ève ne puisse avoir connaissance de ladite clé.

Alice et Bob commencent par se mettre d'accord sur un groupe cyclique $G \cong \mathbb{Z}/N\mathbb{Z}$ (noté multiplicativement) et un générateur $g$ du groupe $G$, ces deux paramètres peuvent être diffusés sur le canal de manière publique sans compromettre la sécurité du protocole.

Alice choisit secrètement un entier $a$ et envoie $g^a$ à Bob. Parallèlement, Bob choisit secrètement un entier $b$ et envoie $g^b$ à Alice. Alice et Bob disposent alors de la clé commune $g^{ab}$. En effet, Alice reçoit $g^b$ de la part de Bob et calcule $(g^b)^a = g^{ab}$; Bob quant à lui reçoit $g^a$ de la part d'Alice et calcule $(g^a)^b = g^{ab}$.

Ni Alice ni Bob ne connaissent la clé secrète de son homologue et pourtant ils ont réussi à se mettre d'accord sur une clé secrète en combinant les deux clés secrètes. Ève qui écoutait le canal de communication dispose de $g^a$ et $g^b$; si elle veut connaitre la clé secrète commune d'Alice et Bob, Ève doit calculer $g^{ab}$ à partir de $g^a$ et $g^b$.

\begin{definition}
On appelle \emph{problème de Diffie-Hellman}, le problème consistant à trouver $g^{ab}$ à partir de $g^a$, $g^b$ et de la base $g$.

Soit $h \in G$. On appelle \emph{logarithme discret de h dans la base g} un élément $a \in \mathbb{Z}/N\mathbb{Z}$ tel que $h=g^a$, on note $\log_g h = a$.
\end{definition}

On remarque que si l'on sait résoudre le problème du logarithme discret alors on sait aussi résoudre le problème de Diffie-Hellman; on calcule le logarithme discret de $g^a$ et $g^b$ dans la base $g$, à savoir $a$ et $b$ puis on calcule $g^{ab}$. Le problème de Diffie-Hellman n'est donc pas plus difficile que le problème du logarithme discret. Il est généralement admis que le problème du logarithme discret et le problème de Diffie-Hellman sont de difficultés équivalentes dans les cas qui nous intéressent; on sait montrer l'équivalence dans quelques cas particuliers \citep{maurer} \citep{muzereau},
mais pas en toute généralité.

\subsection{Logarithme discret}
La \emph{sécurité} du protocole de Diffie-Hellman repose sur le fait que le problème du logarithme discret dans certains groupes est supposé difficile (du point de vue algorithmique). Cette hypothèse n'est pas vérifiée pour tous les groupes que l'on peut considérer; dans $(\mathbb{Z}/N\mathbb{Z},+)$, le problème du logarithme discret est facile. Si $g$ est un générateur de $\mathbb{Z}/N\mathbb{Z}$, on sait qu'il est premier à $N$ et en utilisant l'algorithme d'Euclide étendu on peut calculer un inverse $g^{-1}$ de $g$. Le logarithme discret de $x$ est alors $g^{-1}x$, la complexité du calcul du logarithme discret dans $(\mathbb{Z}/N\mathbb{Z},+)$ est $O(\log^2N)$.

La difficulté du logarithme discret est indépendante du choix d'un générateur. En effet, soit $h,g \in G$ deux générateurs et soit $x \in G$, alors on a :
$$\log_h x = (\log_g x)(\log_g h)^{-1}$$

On en déduit que si l'on sait calculer efficacement le logarithme discret dans la base $g$ alors on sait aussi calculer efficacement le logarithme discret dans la base $h$.

Cependant, la difficulté du logarithme discret n'est pas invariante par isomorphisme. En effet, le calcul du logarithme discret est considérablement plus difficile dans le groupe multiplicatif d'un corps fini $\mathbb{F}^*_q$ que dans le groupe additif $\mathbb{Z}/(q-1)\mathbb{Z}$ qui lui est isomorphe. La raison de cette non-invariance par isomorphisme est que ces isomorphismes ne sont pas, en général, effectivement
calculables. Le calcul de l'isomorphisme entre $\mathbb{Z}/(q-1)\mathbb{Z}$ et $\mathbb{F}^*_q$ est en fait équivalent au problème du logarithme discret dans $\mathbb{F}_q^*$.

La méthode naïve pour résoudre le logarithme discret consiste à tester tous les exposants les uns après les autres; la complexité de cet algorithme est en $O(N)$, ce qui est exponentiel en la taille de $N$ c.-à-d. exponentiel par rapport à $\log N$. Une variante de l'algorithme $\rho$ de \citet{pollard} permet de résoudre le logarithme discret en $O(\sqrt{N})$ opérations. Pour calculer le logarithme discret de $h$ en base $g$ l'algorithme effectue une marche aléatoire sur $(\mathbb{Z}/N\mathbb{Z})^2$ et cherche des collisions de la forme $h^{\alpha_1}g^{\beta_1} = h^{\alpha_2}g^{\beta_2}$ pour en déduire le logarithme discret.

L'algorithme de \citet{pohlig-hellman} permet de réduire la difficulté du logarithme discret dans le cas où l'ordre $N$ du groupe $G$ est composé. L'algorithme consiste à résoudre plusieurs logarithmes discrets dans des groupes plus petits puis à recomposer le logarithme discret dans $G$.

Supposons, pour simplifier et expliquer l'idée générale, que l'ordre $N$ de $G$ est un produit de deux nombres premiers $p$ et $q$ distincts. Si l'on note $G[p]$ (resp. $G[q]$) la composante $p$-primaire (resp. $q$-primaire) de $G$ alors le calcul du logarithme discret dans $G$ se ramène à $2$ calculs de logarithmes discrets dans les sous-groupes primaires par l'application suivante :
\begin{equation*}
\begin{array}{lcl}
G & \longrightarrow & G[p] \times G[q] \\
h & \longmapsto & (h^q, h^p)
\end{array}
\end{equation*}

En supposant que l'on sait calculer les logarithmes discrets $x_1$ (resp. $x_2$) de $h^q$ (resp. $h^p$) dans la base $g^q$ (resp. $g^p$) alors on peut calculer le logarithme discret de $h$ dans le groupe $G$ en utilisant une relation de Bézout entre $p$ et $q$. 

En effet, supposons que l'on ait la relation de Bézout suivante :
$$u_1 q + u_2 p = 1$$

On en déduit alors une expression de $h$ en fonction de $h^p$ et $h^q$ :
\begin{align*}
h &= (h^q)^{u_1}(h^p)^{u_2} \\
  &= (g^q)^{x_1u_1}(g^p)^{x_2u_2}
\end{align*}

Ce qui nous permet de calculer le logarithme discret $x$ de $h$ en base $g$ par la formule suivante :
$$x = x_1 u_1 q + x_2 u_2 p$$

En appliquant simultanément l'algorithme de Pohlig-Hellman pour réduire le logarithme discret dans $G$ en des logarithmes discrets dans des groupes d'ordres premiers et l'algorithme $\rho$ de Pollard pour résoudre le logarithme discret dans les groupes d'ordres premiers; on obtient un algorithme générique qui calcule le logarithme discret en $O(\sum{\alpha_i\sqrt{p_i}}+\log N)$ opérations de groupes.

Si l'on considère que le groupe $G$ est une boîte noire (groupe générique) c.-à-d. que les seules opérations que l'on peut faire sont des multiplications d'éléments de $G$, alors on ne peut pas faire mieux.

\begin{theoreme}[\citet{shoup}]
\label{borneShoup}
Soit $G$ un groupe générique d'ordre $N$, notons $p$ le plus grand facteur premier de $N$. La résolution du problème du logarithme discret dans $G$ nécessite au moins $\Omega(\sqrt{p})$ opérations. 
\end{theoreme}

\subsection{Groupes cryptographiques}
Pour avoir un protocole de Diffie-Hellman sûr, il faut que l'on dispose de groupes dans lesquels le problème du logarithme discret est difficile.

Les premiers groupes qui sont utilisés pour des applications en cryptographie sont les groupes multiplicatifs d'un corps fini $\mathbb{F}^*_q$; ils ont été suggérés par \citet{diffie-hellman}.

On connait des attaques sur le logarithme discret dans ces groupes qui sont plus efficaces que les approches génériques c.-à-d. qui arrivent à battre la borne de Shoup \ref{borneShoup}. Ces attaques (\emph{Index-Calculus}) reposent sur des méthodes qui sont aussi utilisées pour la factorisation d'entiers. La complexité de ces algorithmes est sous-exponentielle $L_q[1/3;c]$ où l'on note
$$L_n[\alpha;c] = \exp(c(\log n)^\alpha(\log \log n)^{1-\alpha})$$

L'utilisation de courbes elliptiques (resp. hyperelliptiques) a été suggérée par \citet{koblitz1,koblitz2}. À l'heure actuelle, on ne connait pas d'algorithme qui arrive à faire mieux que les attaques génériques pour des courbes elliptiques (resp. hyperelliptiques de genre 2) générales; même si l'on connait des attaques spécifiques sur des courbes particulières, comme les courbes elliptiques super-singulières par exemple.

\begin{definition}
Soit $\mathbb{F}_q$ un corps fini de caractéristique impaire.

On appelle \emph{courbe elliptique} (resp. \emph{courbe hyperelliptique de genre 2}) sur $\mathbb{F}_q$ une courbe algébrique projective dont un modèle affine est de la forme :
$$C : y^2 = F(x)$$
où $F(X)$ est un polynôme séparable de degré 3 (resp. de degré 5) à coefficients dans $\mathbb{F}_q$.
\end{definition}

L'ensemble des points dans $\mathbb{F}_q$ d'une courbe elliptique, noté $E(\mathbb{F}_q)$, est naturellement muni d'une structure de groupe abélien. Ce qui permet d'utiliser des méthodes de géométrie algébrique en cryptographie.

L'ensemble des points d'une courbe hyperelliptique de genre 2 n'est, quant à lui, pas naturellement muni d'une structure de groupe. Cependant, si l'on considère des paires de points alors on peut munir cet ensemble d'une structure de groupe; on appelle \emph{Jacobienne} de $C$, noté $\mathcal{J}$, le groupe ainsi construit.

\subsection{Comptage de points}
La difficulté du logarithme discret repose (en partie) sur les propriétés arithmétiques de l'ordre du groupe que l'on considère pour des applications cryptographiques. Dans le cas des courbes elliptiques, l'ordre du groupe sous-jacent correspond au nombre de points sur la courbe elliptique dans $\mathbb{F}_q$; d'où l'importance d'avoir un algorithme efficace pour compter le nombre de points sur une courbe elliptique.

\citet{schoof} publie en 1985 le premier algorithme polynomial de comptage de point sur une courbe elliptique. L'algorithme utilise de manière cruciale les polynômes de divisions et peut être amélioré si l'on trouve une factorisation partielle de ces polynômes; ce qui est fait par \citet{atkin} et \citet{elkies}.

Soit $E$ une courbe elliptique donnée par l'équation de Weierstass courte :
$$E : y^2 = x^3 + ax + b$$

On cherche à calculer le nombre de points sur $\mathbb{F}_p$ de la courbe $E$, que l'on notera $\#E(\mathbb{F}_p)$. Pour ce faire, on va calculer $\#E(\mathbb{F}_p)$ modulo $l$ pour plusieurs nombres premiers $l \neq p$ et utiliser le lemme Chinois pour retrouver $\#E(\mathbb{F}_p)$. D'après la borne de Hasse-Weil \citep{weil}, on a :
$$p+1 - 2\sqrt{p} \leq \#E(\mathbb{F}_p) \leq p+1 + 2\sqrt{p}$$
La largeur de l'intervalle dans la borne de Hasse étant $4\sqrt{p}$; il suffit de calculer $\#E(\mathbb{F}_p) \mod l$ pour suffisamment de $l$ jusqu'à ce que l'on ait l'inégalité suivante :
\begin{equation}
\label{inegaliteSchoof}
\prod_l l > 4\sqrt{p}
\end{equation}

Le morphisme de Frobenius $\pi : E \longrightarrow E$ sur la courbe $E$ est défini en appliquant le Frobenius $x \mapsto x^p$ à chacune des coordonnées; il vérifie une relation quadratique :
$$\pi^2 - t\pi + p = 0$$
où $t$ satisfait l'égalité $\#E(\mathbb{F}_p) = p + 1 - t$.

Pour trouver $t \mod l$, il nous suffit de tester les égalités :
$$\pi^2(x,y) + p(x,y) = t'\pi(x,y), \quad t' = 0,1,...,l-1$$
pour tout point $(x,y)$ de $l$-torsion dans la clôture algébrique $\bar{\mathbb{F}}_p$; cette égalité est valable uniquement pour $t' \equiv t \mod l$.

Le sous-groupe de $l$-torsion
$$E[l] = \{P \in E(\bar{\mathbb{F}}_p), lP = 0\}$$
est défini sur une extension finie de $\mathbb{F}_p$. Cependant, on ne sait pas à priori borner le degré d'une telle extension sans informations supplémentaires.

L'idée de Schoof est de faire les calculs avec un point de $l$-torsion générique. Un polynôme de $l$-division $\Psi_l \in \mathbb{F}_p[X]$ est un polynôme dont les racines sont exactement les coordonnées $x$ des points de $l$-torsion, il est de degré $\frac{l^2-1}{2}$ pour $l \neq 2,p$. On peut calculer les polynômes de division par la relation de récurrence suivante :
\begin{align*}
\Psi_{0} &= 0, \quad \Psi_{1} = 1, \quad \Psi_{2} = 2y \\
\Psi_{2n} &=  \left ( \frac { \Psi_{n}}{2y} \right ) \cdot ( \Psi_{n+2}\Psi^{2}_{n-1} -  \Psi_{n-2} \Psi ^{2}_{n+1})  \quad &(n \geq 1) \\
\Psi_{2n+1} &=  \Psi_{n+2} \Psi_{n}^{ 3}  -  \Psi_{n-1} \Psi ^{3}_{n+1} \quad &(n \geq 1)
\end{align*}

On considère l'algèbre $A_l=\mathbb{F}_p[X,Y]/(\Psi_l(X),Y^2-X^3-aX-b)$; on dira alors que le point $(X,Y)$ de cette algèbre est un point générique de $l$-torsion. En effet, la variété engendrée par l'idéal $(\Psi_l(X),Y^2-X^3-aX-b)$ dans la clôture algébrique est exactement le sous-groupe de $l$-torsion $E[l]$.

En résumé, l'algorithme de Schoof consiste à tester les égalités :
\begin{equation}
\label{polCarFrob}
(X^{p^2},Y^{p^2}) + (p \mod l)(X,Y) = t'(X^p,Y^p), \quad t' = 0,1,...,l-1
\end{equation}
modulo l'idéal $(\Psi_l(X),Y^2-X^3-aX-b)$ pour trouver $t \mod l$. On fait ce calcul pour suffisamment de $l$ et on applique le lemme Chinois pour retrouver $t$.

L'algorithme nécessite le calcul de $X^p$, $X^{p^2}$, etc. dans l'algèbre $A_l$ et au plus $2l$ additions de points. Les calculs dans l'algèbre $A_l$ se font (du moins pour la coordonnée X) en réduisant modulo $\Psi_l(X)$; la multiplication de deux polynômes modulo $\Psi_l(X)$ s'effectue en $O(l^4)$ multiplications dans $\mathbb{F}_p$, ce qui demande $O((l^2\log p)^2)$ opérations élémentaires. Les calculs de $X^p$, $X^{p^2}$, etc. se font en $O((l^2\log p)^2\log p)$ opérations; tandis que le total des additions de points se fait en $O(l(l^2\log p)^2)$ opérations.

Le théorème des nombres premiers montre que l'on a $l = O(\log p)$. En effet, le théorème des nombres premiers est équivalent à l'assertion :
$$\sum_{l \leq x}\log l = x + o(x)$$

D'après l'inégalité \ref{inegaliteSchoof}, on en déduit bien que $l = O(\log p)$. Ceci permet de conclure que la complexité de l'algorithme de Schoof est $O(\log^8 p)$.

\section{Courbes de genre 2}

L'utilisation des courbes elliptiques en cryptographie connait une montée en puissance depuis son introduction par Koblitz en 1985, l'avantage principal de cette approche est que les clés sont de tailles plus petites ce qui est très apprécié pour des systèmes embarqués dont la mémoire est limité. Jusqu'à récemment, les records de rapidité concernant le protocole de Diffie-Hellman ont été obtenus grâce à une arithmétique rapide sur les courbes elliptiques.

Les courbes hyperelliptiques de genre $> 2$ offrent une généralisation naturelle pour des applications en cryptographie; cependant, il existe des attaques sous-exponentielles au logarithme discret dans les Jacobiennes de ces courbes \citep{gaudry2} ce qui nous oblige à compenser par des paramètre plus grand.

À l'heure actuelle, on ne connait aucune attaque sous-exponentielle au logarithme discret sur les Jacobiennes de courbes de genre 2; ce qui explique (du moins en partie) leur intérêt en cryptographie.

D'autre part, \citet{gaudry} présente un algorithme de multiplication scalaire sur la surface de Kummer d'une courbe de genre 2 qui est compétitif par rapport à la multiplication scalaire sur une courbe elliptique.

\subsection{Définition}
Soit $k$ un corps de caractéristique impaire.

\begin{definition}
Une courbe algébrique $C$ définie sur $k$ est dite \emph{hyperelliptique de genre 2} si elle admet une équation de la forme :
$$C : y^2 = F(x)$$
où $$F(X) = f_0 + f_1X + ... + f_6X^6 \in k[X]$$
est un polynôme séparable de degré 6.
\end{definition}

\begin{remarque}
On se permet un abus de notation dans la définition précédente; en effet, on dit que la courbe $C$ est d'équation $y^2 = F(x)$ alors qu'on devrait plutôt considérer la courbe projective associée.

Si l'on considère l'équation homogène associée :
$$Y^2Z^4 = \mathbf{F}(X,Z)$$
alors le point à l'infini $Z=0$ est un point singulier; pour obtenir une courbe lisse, on effectue un éclatement au point singulier. La courbe que l'on obtient présente alors deux points à l'infini $\infty_1$ et $\infty_2$.

Un modèle projectif non-singulier dans $\mathbb{P}^4$ de la courbe $C$ est donné par \citep{cassels-Flynn} :
\begin{align*}
Y^2 &= f_0X_0^2 + f_1X_0X_1 + f_2X_1^2 + f_3X_1X_2 + f_4X_2^2 + f_5X_2X_3 + f_6X_3^2 \\
0 &= X_0X_2 - X_1^2 \\
0 &= X_0X_3 - X_1X_2 \\
0 &= X_1X_3 - X_2^2
\end{align*}
Les points correspondant à l'équation affine ($X_0 \neq 0$) sont obtenus en posant :
$$X_j = X^j, \quad Y = Y$$
\end{remarque}

\begin{remarque}
On peut montrer que toute courbe de genre 2 est birationnelle à une courbe hyperelliptique de genre 2 \citep{cassels-Flynn}. Lorsque l'on parlera de courbe de genre 2, on sous-entend que l'on considère une équation sous forme hyperelliptique; autrement dit une équation de la forme $y^2 = F(x)$.
\end{remarque}

Une telle courbe est munie d'une \emph{involution hyperelliptique}
\begin{equation*}
\begin{array}{lrcl}
\iota :&C & \longrightarrow & C \\
& (x,y) & \longmapsto & (x,-y)
\end{array}
\end{equation*}

Le point $(x,-y)$ est appelé l'opposé de $(x,y)$. Il y a 6 points particuliers sur la courbe qui sont invariants par l'involution, on les appelle \emph{points de Weierstrass}; ils sont de la forme $(\theta,0)$ où $\theta$ est l'une des 6 racines de $F(X)$.

Dans la suite, on s'intéressera à des fonctions (rationnelles) sur la courbe $C$ à valeurs dans $k$.
\begin{definition}
On dira que $f : C \subset k^2 \longrightarrow k$ est une \emph{fonction} sur $C$ si elle est une fraction rationnelle $f = \frac{P}{Q}$; où $P(X,Y)$ et $Q(X,Y)$ sont des polynômes à coefficients dans $k$ avec $Q \not\equiv 0$ sur $C$.
\end{definition}

En utilisant l'équation de la courbe, on peut remplacer toutes les apparitions de $y^2$ dans $P$ et $Q$ par l'expression $F(x)$; on en déduit que les fonctions sur $C$ sont de la forme :
$$\frac{P_1(x) + P_2(x)y}{Q_1(x) + Q_2(x)y}$$
où $P_1(X),P_2(X),Q_1(X),Q_2(X)$ sont des polynômes à coefficients dans $k$.

\begin{definition}
L'ensemble des fonctions (rationnelles) sur $C$ est appelé \emph{corps de fonctions}, on le note $k(C)$.
\end{definition}

On peut aussi définir le corps de fonctions comme le corps des fractions de l'anneau des coordonnées :
\begin{align*}
k(C) &= k(x)[y]/(y^2-F(x)) \\
  &= Frac(k[x,y]/(y^2-F(x)))
\end{align*}
\subsection{Diviseurs et Jacobienne}

\begin{definition}
Un \emph{diviseur} $D = \sum n_x x$ est une somme formelle finie où $x \in C$ et $n_x \in \mathbb{Z}$; dit autrement, un diviseur est un élément du groupe abélien libre $Div$ engendré par les points de $C$.

On dit que $n_x$ est la \emph{multiplicité} du point $x$ et l'on a $n_x = 0$ sauf pour un nombre fini de points $x \in C$. On appelle \emph{degré} de $D$, noté $deg(D)$, la somme des multiplicités des points de $C$. Pour finir, le \emph{support} de $D$ est l'ensemble des $x \in C$ tel que $n_x$ est non nul.
\end{definition}

Une fonction non nulle $f$ sur $C$ détermine un diviseur noté $div(f)$; la multiplicité en un point $x \in C$ du diviseur $div(f)$ est égale à l'ordre du zéro en $x$ de $f$ ou à l'opposé de l'ordre du pôle en $x$. Un diviseur de la forme $div(f)$ est dit \emph{principal}, un tel diviseur est de degré 0 \citep{menezes}.

Les diviseurs principaux forment un sous-groupe de $Div$, noté $Pr$.
\begin{definition}
On considère le quotient du groupe des diviseurs par les diviseurs principaux $Pic = Div/Pr$ et on l'appelle \emph{groupe de Picard}.
La Jacobienne de la courbe $C$, notée $\mathcal{J}$, est le sous-groupe des éléments du groupe de Picard de degré 0; que l'on note aussi $Pic^0$.
\end{definition}

On dira que deux diviseurs sont \emph{linéairement équivalents} s'ils sont dans la même classe; autrement dit, s'ils diffèrent d'un diviseur principal.

\begin{remarque}
Dans le cas des courbes elliptiques, la courbe est munie naturellement d'une structure de groupe; ce n'est plus le cas en genre $2$. Cependant, la Jacobienne étant munie d'une structure de groupe sous-jacente, on remarque que les 2-points forment un groupe sur une courbe de genre $2$.
\end{remarque}

\begin{remarque}
La Jacobienne $\mathcal{J}$ est munie d'une structure de variété algébrique; la construction consiste essentiellement à prendre 2 copies de $C$ que l'on recolle le long du graphe de l'involution hyperelliptique \citep{cassels-Flynn}.
\end{remarque}

Soit $P=(x_0,y_0) \in C$, les points $P$ et $\iota(P)$ sont les zéros de la fonction $(x-x_0)$ qui a deux pôles à l'infini $\infty_1$ et $\infty_2$; autrement dit, on a :
$$div(x-x_0) = P + \iota(P) = \infty_1 + \infty_2$$

On en déduit que toute classe de diviseur $D$ dans la Jacobienne $\mathcal{J}$ se représente sous la forme :
$$D = \sum_{i=1}^r{P_i} - r\infty_2$$
où $P_i \neq \infty_2$ et $P_j \neq \iota(P_i)$ pour $i \neq j$. Un diviseur sous cette forme sera appelé \emph{semi-réduit}.

Le théorème de \emph{Riemann-Roch} permet de représenter un élément de la Jacobienne par un diviseur semi-réduit avec la condition supplémentaire $r \leq 2$; on dira alors que le diviseur est \emph{réduit}.
\begin{theoreme}[Riemann-Roch]
Soit $D$ un diviseur sur $C$, l'ensemble des fonctions $f$ tel que $div(f) \geq -D$ auquel on rajoute la fonction nulle forme un espace vectoriel noté $L(D)$.

Il existe un diviseur $K$, appelé diviseur canonique, tel que
$$l(D) = dim (L(D)) = deg(D) - 1 + dim (L(K - D))$$
\end{theoreme}

D'après le théorème de Riemann-Roch, on a :
$$l(D+2\infty_2) = 1 + l(K-D)$$

On en déduit alors que $l(D+2\infty_2) \geq 1$ car $l(K-D) \geq 0$; donc il existe une fonction $f$ non nulle telle que $div(f) \geq -D - 2\infty_2$.

On pose $D' = div(f) + D + 2\infty_2 \geq 0$, alors on a :
$$D' - 2\infty_2 \sim D$$
De plus, $deg(D') = 2$; donc il existe $P_1, P_2 \in C$ tel que :
$$D' = P_1 + P_2$$

Pour obtenir un diviseur réduit, il nous suffit de mettre $D = P_1 + P_2 - 2\infty_2$ sous forme semi-réduite; ce qui conserve la condition $r \leq 2$. On en déduit alors le corollaire suivant :

\begin{corollaire}
\label{div_red}
Tout point de la Jacobienne $\mathcal{J}$ est représenté par un diviseur réduit.
\end{corollaire}

\subsection{Applications en cryptographie}
D'après les bornes de Hasse-Weil, l'ordre de grandeur du nombre de points de la Jacobienne d'une courbe de genre 2 est :
$$\#\mathcal{J}(\mathbb{F}_p) \approx p^2$$

L'idée d'utiliser des courbes de genre $2$ vient du fait que si l'on veut un groupe du même ordre de grandeur que pour une courbe elliptique, alors on peut utiliser un corps fini de taille plus petite. Par exemple, pour avoir un groupe d'ordre approximativement $2^{256}$, dans le cas d'une courbe elliptique il nous faut $log_2(p) \approx 256$; alors que pour une courbe de genre 2 il nous faut seulement $log_2(p) \approx 128$.

L'ordre du groupe $N$ mesure la \emph{sécurité} que l'on peut espérer en utilisant ce groupe dans un protocole cryptographique.
En effet, en supposant que le groupe est une boite noire pour lequel on n'a aucune information supplémentaire; alors le meilleur algorithme connu pour résoudre le logarithme discret est en $O(log(N)+\sqrt{p_1})$, où $p_1$ est le plus grand facteur premier dans l'ordre du groupe.

On voit donc que le temps nécessaire pour casser un protocole à base de courbes hyperelliptiques dépend uniquement (d'un point de vue heuristique) du plus grand facteur premier de l'ordre du groupe, c'est pour cette raison que l'on s'intéresse le plus souvent à des groupes dont l'ordre est proche d'être un nombre premier (c.-à-d. $N/p_1$ est petit).

Il y a un compromis à faire entre la complexité des formules d'addition et la taille du corps fini sous-jacent, on espère ainsi avoir une \emph{arithmétique plus efficace} en genre 2 comparée aux courbes elliptiques à un niveau de sécurité fixe.
 
\subsection{Représentation de Mumford}
Soit $D = \sum_{i=1}^r{P_i} - r\infty_2$ un diviseur réduit sur $C$, alors $D$ peut se représenter de manière unique par deux polynômes $(u,v)$ à coefficients dans $k$ tel que :
\begin{itemize}
\item $u(x) =  \prod_{i=1}^r (x - x(P_i) )$,
\item $deg(v) < 2$,
\item $v(x(P_i)) = y(P_i)$ pour tout $1 \leq i \leq r$,
\item $u$ divise $v^2 - f$
\end{itemize}
On appelle $(u,v)$ la représentation de Mumford du diviseur $D$.

Le théorème suivant nous assure qu'il y a bien une correspondance biunivoque entre les points de la Jacobienne et les représentations de Mumford :

\begin{theoreme}[Mumford]
Toute paire de polynômes $(u,v)$ à coefficients dans $k$ vérifiant ces conditions est la représentation de Mumford d'un diviseur réduit sur $C$.
\end{theoreme}

On donne maintenant une interprétation géométrique de la loi de groupe sur la Jacobienne $\mathcal{J}$. Soit $D_1 = P_1 + P_2 - 2\infty_2$ et $D_2 = P_3 + P_4 - 2\infty_2$. Pour calculer $D_1 + D_2$, on commence par trouver le polynôme $Q(X)$ (par interpolation de Lagrange) de degré $3$ passant par $P_1,P_2,P_3,P_4$. La courbe d'équation $y = Q(x)$ coupe la courbe $C$ en $2$ autres points $P_5$ et $P_6$.

On note $D_3 = P_5 + P_6 - 2\infty_2$, alors on a :
$$D_1 + D_2 + D_3 = div(P) = 0$$

On en déduit que :
$$D_1 + D_2 = -D_3 = \iota(P_5) + \iota(P_6) - 2\infty_2$$

L'algorithme de \citet{cantor} explicite la loi de groupe sur la Jacobienne pour des points dans leurs représentations de Mumford.

\section{Arithmétique de genre 2 selon Cassels et Flynn}
On développe dans cette section la théorie de Cassels et Flynn sur l'arithmétique des courbes de genre 2 et des surfaces de Kummer. On mettra l'accent sur l'aspect calculatoire, l'objectif étant d'avoir des formules explicites de pseudo-addition et de doublement.

Les formules que Cassels et Flynn obtiennent sont relativement lourdes, l'avantage étant que l'on ne fait aucune restriction sur les courbes de genre 2 que l'on considère. On verra plus loin l'approche de Gaudry qui se base sur une paramétrisation par des fonctions thêtas, on obtient ainsi des formules relativement simple et très rapide.

Soit $k$ un corps de caractéristique impaire, soit $C$ une courbe de genre 2 sous la forme :
$$C : y^2 = F(x)$$
où $$F(X) = f_6X^6 + f_5X_5 + ... + f_0$$ est un polynôme séparable de degré 6.

\subsection{Surface de Kummer}
Les surfaces de Kummer ont été introduites par Kummer en 1864 dans le cadre de recherches sur la diffraction de la lumière. Sur $\mathbb{C}$ ce sont des quartiques qui possèdent 16 points singuliers; c'est le nombre maximal de points singuliers pour une telle quartique qui ne contient pas de courbe singulière.

Pour nous, les surfaces de Kummer seront un moyen d'avoir une arithmétique plus rapide que sur la Jacobienne au prix d'une légère perte d'information.

\begin{definition}
La \emph{surface de Kummer} $\mathcal{K}$ de la courbe $C$ est l'image de l'application suivante :
\begin{equation*}
\begin{array}{lrcl}
\xi : &\mathcal{J} & \longrightarrow & \mathbb{P}^3(k) \\
& \{(x,y),(u,v)\} & \longmapsto & (1,x+u,xu,\beta_0(x,y,u,v))
\end{array}
\end{equation*}
où l'on note
$$\beta_0(x,y,u,v) = \frac{F_0(x,u)-2yv}{(x-u)^2}$$
et
\begin{align*}
F_0(x,u)=2f_0&+f_1(x+u)+2f_2xu+f_3(x+u)xu \\
&+2f_4(xu)^2+f_5(x+u)(xu)^2+2f_6(xu)^3.
\end{align*}
\end{definition}

\begin{remarque}
La paramétrisation de la surface de Kummer semble un peu arbitraire. On peut tout de même remarquer que les trois premières coordonnées permettent de reconstruire le premier polynôme dans la représentation de Mumford. La dernière coordonnée quand à elle donne une information sur le produit $yv$ et permet de retrouver une partie du deuxième polynôme dans la représentation de Mumford. Avec un peu de calcul, on arrive à calculer $(y+v)^2$ et l'on a juste un choix de signe à faire pour retrouver complètement la représentation de Mumford.
\end{remarque}

Soit $P \in \mathcal{J}$, on note $\xi_1(P),...,\xi_4(P)$ les coordonnées homogènes de l'image $\xi(P)$ du point $P$ dans la surface de Kummer.

\begin{remarque}
Dans le cas des courbes elliptiques, la Jacobienne est isomorphe à la courbe elle-même et la variété de Kummer est simplement la droite projective que l'on considère en tant que coordonnée $X$.
\end{remarque}

La surface de Kummer est une variété projective définie par une équation homogène de degré 4 de la forme :
$$K(\xi_1,\xi_2,\xi_3,\xi_4)=K_2\xi_4^2 + K_1\xi_4 + K_0 = 0$$
où les formules pour $K_0$, $K_1$ et $K_2$ sont données par \citet{eqKum}.

Deux points opposés $P$ et $-P$ dans $\mathcal{J}$ sont envoyés sur le même point dans $\mathcal{K}$. Les 16 points de 2-torsion de la Jacobienne sont leurs propres opposés et sont envoyés sur les 16 points singuliers de la surface de Kummer. La Jacobienne est un revêtement de degré 2 de la surface de Kummer ramifié aux points singuliers.

L'image de l'élément neutre de $\mathcal{J}$ a pour image le point $\mathbf{o} = (0,0,0,1)$ dans $\mathcal{K}$. Les autres points de 2-torsion de $\mathcal{J}$ sont obtenus en prenant deux points de Weierstrass distincts; c'est à dire de la forme $\{(\theta_1,0),(\theta_2,0)\}$ où $\theta_1$ et $\theta_2$ sont deux racines distinctes du polynôme $F(X)$. On trouve ainsi 15 points de 2-torsion non triviaux qui forment avec l'élément neutre la totalité de la 2-torsion de la Jacobienne $\mathcal{J}$.

On va maintenant calculer l'image de la 2-torsion dans $\mathcal{K}$; en effet, si l'on écrit le polynôme $F(X)$ sous la forme :
$$F(X) = (X-\theta_1)(X-\theta_2)H(X)$$
où $H(X)$ est un polynôme de degré 4 et l'on note :
$$H(X) = h_0 + h_1X + ... + h_4X^4$$

Alors un calcul montre que l'on a l'égalité suivante :
$$\beta_0((\theta_1,0),(\theta_2,0)) = -h_0 -h_2\theta_1\theta_2 - h_4(\theta_1\theta_2)^2$$

On en déduit que l'image du point de 2-torsion $\{(\theta_1,0),(\theta_2,0)\}$ dans $\mathcal{K}$ est :
$$\xi((\theta_1,0),(\theta_2,0)) = (1,\theta_1+\theta_2,\theta_1\theta_2, -h_0 -h_2\theta_1\theta_2 - h_4(\theta_1\theta_2)^2)$$

Sur la surface de Kummer $\mathcal{K}$, on ne dispose pas d'une structure de groupe bien définie puisque l'on a quotienté la Jacobienne $\mathcal{J}$ par un sous-ensemble qui n'est pas un sous-groupe.

Cependant, $\mathcal{J}$ est munie d'une structure de groupe et l'on peut donc définir une opération qui ressemble très fortement à une addition, que l'on appellera \emph{pseudo-addition}, sur la surface de Kummer.

\subsection{Addition par un point de 2-torsion}
\label{add2tors}
La surface de Kummer $\mathcal{K}$ identifie deux points opposés $P$ et $-P$; donc si l'on veut additionner $\pm P$ et $\pm Q$, on ne peut pas faire la différence entre $\pm (P+Q)$ et $\pm (P-Q)$. Cependant, pour un point de 2-torsion, on a l'égalité $-Q=Q$; donc l'addition $P+Q$ est bien définie sur $\mathcal{K}$. La translation par un point de 2-torsion est en fait une application linéaire de $\mathbb{P}^3$.

Un point de 2-torsion non trivial est donné par un facteur de degré 2 du polynôme $F(X)$; dans le cas où le polynôme $F(X)$ est irréductible alors la Jacobienne $\mathcal{J}$ ne contient pas de point de 2-torsion rationnel non trivial.

Supposons que l'on dispose d'une factorisation du polynôme $F(X)$ sous la forme :
$$F(X) = G(X)H(X)$$
où $G(X)$ est un polynôme de degré 2, $H(X)$ est un polynôme de degré 4 et l'on note :
\begin{align*}
G(X) &= g_0 + g_1X + g_2X^2 \\
H(X) &= h_0 + ... + h_4X^4
\end{align*}

La matrice $W$ de l'application linéaire définie par l'action de translation par le point de 2-torsion correspondant au polynôme $G(X)$ est donnée par \citet{eqKum} en fonction des coefficients des polynômes $G(X)$ et $H(X)$.

\[ \left(
  \begin{array}{ c c c c}
     g_2^2h_0 + g_0g_2h_2 - g_0^2h_4 & g_0g_2h_3 - g_0g_1h_4 & g_1g_2h_3 - g_1^2h_4 + 2g_0g_2h_4  & g_2  \\
     -g_0g_2h_1 - g_0g_1h_2 + g_0^2h_3 & g_2^2h_0 - g_0g_2h_2 + g_0^2h_4  & g_2^2h_1 - g_1g_2h_2 - g_0g_2h_3  & -g_1  \\
     -g_1^2h_0 + 2g_0g_2h_0 + g_0g_1h_1 & -g_1g_2h_0 + g_0g_2h_1  & -g_2^2h_0 + g_0g_2h_2 + g_0^2h_4  & g_0 \\
     w_{41} & w_{42}   & w_{43}  & w_{44}  \\
  \end{array} \right)
\]

La translation par un point de 2-torsion étant une involution, on en déduit que $W^2$ est une homothétie; cette condition nous permet de trouver les coefficients $w_{41},w_{42} ,w_{43},w_{44}$ manquants, on trouve :
\begin{align*}
w_{41} &= -g_1g_2^2h_0h_1 + g_0g_2^2h_1^2 + g_1^2g_2h_0h_2 - 4g_0g_2^2h_0h_2 \\
    &   -g_0g_1g_2h_1h_2 + g_0g_1g_2h_0h_3 - g_0^2g_2h_1h_3      \\
w_{42} &= g_1^2g_2h_0h_3 - 2g_0g_2^2h_0h_3 - g_0g_1g_2h_1h_3 - g_1^3h_0h_4 \\
    &+ 4g_0g_1g_2h_0h_4 + g_0g_1^2h_1h_4 - 2g_0^2g_2h_1h_4     \\
w_{43} &= -g_0g_2^2h_1h_3 - g_0g_1g_2h_2h_3 + g_0^2g_2h_3^2 + g_0g_1g_2h_1h_4 \\
    & + g_0g_1^2h_2h_4 - 4g_0^2g_2h_2h_4 - g_0^2g_1h_3h_4        \\        
w_{44} &= -g_2^2h_0 - g_0g_2h_2 - g_0^2h_4.
\end{align*}

De plus, la constante $c$ telle que $W^2 = cI$ est égale au résultant de $G(X)$ et de $H(X)$ comme le montre un \href{run:./add2tors.sag}{calcul formel}.

\subsection{Pseudo-addition et doublement}
Le théorème suivant permet de calculer le point $\pm (P+Q)$ à partir des points $\pm P$,$\pm Q$ et $\pm(P-Q)$. On appellera cette opération la \emph{pseudo-addition} de $P$ et $Q$; sous entendu, sachant que l'on se donne le point $\pm(P-Q)$.

\begin{theoreme}[\citet{cassels-Flynn}]
Soit $P,Q \in \mathcal{K}$. Il existe des polynômes $B_{ij}$ biquadratiques tels que projectivement on ait les relations :
$$(\xi_i(P+Q)\xi_j(P-Q) + \xi_i(P-Q)\xi_j(P+Q)) = (2B_{ij}(\xi(P),\xi(Q))$$
\end{theoreme}

Les polynômes $B_{ij}$ sont calculables explicitement et l'on trouvera les expressions sur la page de \citet{biquad}. La méthode pour calculer lesdits polynômes $B_{ij}$ consiste d'une part à remarquer que l'on sait calculer cette expression dans le cas où l'un des points est de 2-torsion et ensuite à montrer qu'il existe un unique polynôme qui se spécialise en donnant cette expression sur la 2-torsion; on renvoie à \citep{cassels-Flynn} pour plus de détails.

Une fois que l'on a défini la pseudo-addition, on peut définir un doublement en prenant $P=Q$; dans ce cas $P-Q = 0$ dans $\mathcal{J}$ qui est envoyé sur $(0,0,0,1)$ dans $\mathcal{K}$. On en déduit les relations de doublement suivantes :
$$\xi_4(2P) = B_{44}(\xi(P),\xi(P))$$
et
$$\xi_i(2P) = 2B_{i4}(\xi(P),\xi(P)), \quad (1\leq i \leq 3)$$

Une fois que l'on sait calculer le doublement $2P$, on peut calculer par récurrence la multiplication scalaire $nP$ pour tout $n\in\mathbb{N}$. Pour ce faire, on suppose par récurrence que l'on connait $(n-1)P$ et $(n-2)P$; on calcule alors la pseudo-addition de $(n-1)P$ et $P$ (la différence étant $(n-2)P$), ce qui donne bien le point $nP$ que l'on cherche à calculer.

\subsection{Courbes de Montgomery en genre 2 selon Duquesne}
Les formules de pseudo-addition et de doublement qu'obtiennent Cassels et Flynn ne sont pas adaptées à des applications en cryptographie. Ces formules ne sont pas compétitives comparées à l'arithmétique sur les courbes de Montgomery.

\citet{duquesne} donne un analogue des courbes sous forme de Montgomery en genre 2 et utilise ces courbes pour avoir une arithmétique relativement rapide.

\begin{definition}[\citep{duquesne}]
Une courbe $C$ de genre 2 sous forme de Montgomery est une courbe lisse d'équation :
$$C : By^2 = x^5 + f_4x^4 + f_3x^3 + f_2x^2 + x.$$
\end{definition}

\begin{remarque}
Comme Ben me le fait remarquer, malgré l'apparente similarité avec les courbes de Montgomery, qui sont des courbes elliptiques dont l'équation est de la forme
$$E : By^2 = x^3 + Ax^2 + x,$$
la définition de Dusquesne ne partage pas les mêmes propriétés que celles-ci.

En effet, les courbes de Montgomery disposent d'une 2-isogénie, ce qui est un élément crucial pour avoir une arithmétique rapide; en ce qui concerne les courbes de Dusquesne, elles ne disposent pas forcément d'une (2,2)-isogénie (pour que ce soit le cas, il faudrait que le polynôme $X^4+f_4X^3+f_3X^2+f_2X+1$ se décompose en produit de deux polynômes quadratiques). Notre but dans la suite va être de donner des conditions sur la courbe pour quelle dispose d'une (2,2)-isogénie et que celle-ci soit relativement rapide à calculer.
\end{remarque}

Comme pour les surfaces de Kummer dans \citep{cassels-Flynn}, on obtient des biquadratiques qui permettent de calculer la pseudo-addition et le doublement; donc par récurrence la multiplication scalaire.

L'avantage de ces formules (à celles de Cassels et Flynn) est que l'on obtient une arithmétique légèrement plus rapide tout en gardant un espace de courbe suffisamment grand pour des applications cryptographiques.

\begin{proposition}[\citet{duquesne}]
\label{propDuquesne}
Soit $C$ une courbe de genre 2 sous forme de Montgomery. Le calcul d'une pseudo-addition nécessite 31 multiplications et 2 élévations au carré. Le calcul d'un doublement nécessite 31 multiplications et 5 élévations au carré.
\end{proposition}

La méthode de Montgomery pour le calcul de la multiplication scalaire dans une courbe elliptique consiste à effectuer les calculs sur la variété de Kummer (qui correspond à la coordonnée X). Il s'agit d'une variante de la méthode d'exponentiation rapide par doublement successif; cependant, on doit faire un peu plus de calcul à chaque itération à cause du fait que l'on a perdu une partie de l'information sur le groupe de départ (qui n'est autre que la Jacobienne $\mathcal{J}$).

L'algorithme \ref{algoMontgomery} donne le pseudo-code de la méthode de Montgomery générale sans se préoccuper de l'implémentation exacte de l'arithmétique sur $\mathcal{K}$; on utilise les primitives suivantes dans l'algorithme :
\begin{itemize}
\item $DOUBLE(P)$ : retourne le doublement du point $P$.
\item $ADD(P,Q,R)$ : si $R = P-Q$, retourne le point $P+Q$.
\item $TAILLE(n)$ : retourne la taille de l'entier $n$; c'est à dire $\left \lceil{\log_2(n)}\right \rceil$.
\item $BIT(i,n)$ : retourne le $i$-ième bit du nombre $n$.
\end{itemize}

\begin{algorithm}
\label{algoMontgomery}
\SetKwInOut{Input}{Entrée}\SetKwInOut{Output}{Sortie}
\caption{Algorithme de Montgomery}
\Input{ Un point $P \in \mathcal{K}$ et un entier $n > 1$. }
\Output{ Le point $nP$. }
\textbf{Initialisation} : $(P_1,P_2) = (P,DOUBLE(P))$
\BlankLine
\For{i de TAILLE(n)-1 à 0}{
 \If{BIT(i,n) = 0}{$(P_1,P_2) = (DOUBLE(P_1),ADD(P_1,P_2,P))$}
 \Else{$(P_1,P_2) = (ADD(P_1,P_2,P),DOUBLE(P_2))$}
}
\Return{$P_1$}
\end{algorithm}

Si l'on utilise l'algorithme \ref{algoMontgomery} sur une courbe de genre 2 sous forme de Montgomery, on obtient le résultat suivant :
\begin{theoreme}[\citet{duquesne}]
\label{thduquesne}
Soit $C$ une courbe de genre 2 sous forme de Montgomery et $\mathcal{K}$ la surface de Kummer correspondante.

Le calcul de la multiplication scalaire par $n > 1$ dans $\mathcal{K}$ nécessite $62\log_2 n$ multiplications et $7\log_2 n$ élévations au carré.

De plus, si la courbe $C$ a une équation de la forme :
$$By^2 = x^5 + f_3 x^3 + \epsilon x^2 + x$$
où $\epsilon \in \{0,\pm 1\}$, alors la multiplication scalaire par $n > 1$ dans $\mathcal{K}$ nécessite $46\log_2 n$ multiplications et $6\log_2 n$ élévations au carré.
\end{theoreme}

\begin{remarque}
Comme peut le laisser sous-entendre ce théorème, on voit que l'apparente rapidité de l'arithmétique vient finalement du fait que certains coefficients sont nuls ou suffisamment simples pour que les formules d'addition et de duplication se simplifient légèrement.
\end{remarque}

Un avantage supplémentaire de la méthode de Montgomery pour la cryptographie est que son utilisation permet de se prémunir contre des attaques de type \emph{side-channel}; où l'on analyse la consommation en énergie ou le temps d'exécution pour en déduire une information sur la clé secrète. Cette approche est donc intéressante pour des applications cryptographiques sur des cartes à puces.

\section{Arithmétique rapide de genre 2}

Suivant une suggestion de \cite{chudnovsky}, \citep{gaudry} présente des formules de doublement et de pseudo-addition provenant de la théorie des fonctions thêtas. Ces formules permettent d'obtenir l'algorithme de multiplication scalaire le plus rapide à l'heure actuelle.

Comme dans le théorème \ref{thduquesne}, on utilise l'algorithme de Montgomery \ref{algoMontgomery} qui permet d'effectuer une multiplication scalaire en utilisant seulement un doublement et une pseudo-addition par bit; en particulier, on n'a pas besoin de la structure complète de groupe et l'on effectue les calculs sur la surface de Kummer.

Ces formules font apparaitre des symétries qui se reflètent sur la géométrie de la courbe; en particulier, la 2-torsion est rationnelle. Ceci laisse penser que toutes les courbes de genre 2 ne sont pas forcément adaptées pour l'utilisation de telles formules rapides. On s'intéressera dans une section suivante à des conditions suffisantes sous lesquelles une courbe de genre 2 peut être mise sous cette forme et dispose donc d'une arithmétique rapide. 

\subsection{Fonctions thêtas}
Les \emph{fonctions thêtas de Riemann} sont une famille de fonctions holomorphes indexées par le \emph{demi-espace de Siegel} $\mathfrak{H}_2$ des matrices de $M_2(\mathbb{C})$ tels que la partie imaginaire soit une matrice définie positive. Les fonctions thêtas sont alors définies comme des translatés à un facteur exponentiel prés des fonctions thêtas de Riemann.

\begin{definition}
Soit $\Omega$ une matrice du demi-espace de Siegal $\mathfrak{H}_2$, la fonction thêta de Riemann $\theta(.,\Omega)$ associée à $\Omega$ est la fonction holomorphe de $\mathbb{C}^2$ dans $\mathbb{C}$ définie par :
$$\theta(z,\Omega) = \sum_{n \in \mathbb{Z}^2}{\exp(i\pi n^T \Omega n + 2i\pi n^T z)}$$
pour tout $z \in \mathbb{C}^2$.

On vérifiera que la série est bien convergente grâce à la condition sur la matrice $\Omega$; ce qui permet de conclure que la fonction $\theta$ est bien holomorphe.

Soit $a,b \in \mathbb{Q}^2$, la \emph{fonction thêta de caractéristique} $(a,b)$ est définie par :
$$\theta[a;b](z,\Omega) = \exp(i\pi a^T\Omega a + 2i\pi a^T(z+b))\theta(z + \Omega a + b, \Omega)$$
\end{definition}

On fixe la matrice $\Omega$ et l'on numérote les fonctions thêtas qui nous seront utiles dans la suite; suivant \citet{gaudry} on pose :
\begin{align*}
\theta_1(z) &= \theta[(0,0);(0,0)](z,\Omega) \\
\theta_2(z) &= \theta[(0,0);(\frac{1}{2},\frac{1}{2})](z,\Omega) \\
\theta_3(z) &= \theta[(0,0);(\frac{1}{2},0)](z,\Omega) \\
\theta_4(z) &= \theta[(0,0);(0,\frac{1}{2})](z,\Omega)
\end{align*}

De plus, on appellera \emph{thêtas constantes} l'évaluation de ces fonctions thêtas au point $z=(0,0)$; on les note respectivement $a,b,c,d$.

\subsection{Surface de Kummer}

\begin{definition}
La surface de Kummer $\mathcal{K}$ associée à la matrice $\Omega$ est définie comme l'image de l'application suivante :
\begin{equation*}
\begin{array}{lrcl}
\phi :&\mathbb{C}^2 & \longrightarrow & \mathbb{P}^3(\mathbb{C}) \\
& z & \longmapsto & (\theta_1(z),\theta_2(z),\theta_3(z),\theta_4(z))
\end{array}
\end{equation*}
\end{definition}

La surface de Kummer $\mathcal{K}$ est une variété projective définie par une équation homogène de degré 4 de la forme suivante \citep{gaudry} :
\begin{align*}
X^4+Y^4+Z^4+T^4&+2EXYZT-F(X^2T^2+Y^2Z^2) \\
&-G(X^2Z^2+Y^2T^2)-H(X^2Y^2+Z^2T^2)=0
\end{align*}
où l'on note $(X:Y:Z:T)=(\theta_1(z):\theta_2(z):\theta_3(z):\theta_4(z))$ les coordonnées projectives sur $K$; les constantes $E,F,G,H$ étant des fonctions rationnelles de $a,b,c,d$.

L'application $\phi$ est $(\mathbb{Z}^2 + \Omega\mathbb{Z}^2)$-périodique, elle définit donc une application sur le tore complexe $\mathbb{C}^2/(\mathbb{Z}^2+\Omega\mathbb{Z}^2)$. Cependant, $\phi$ n'est pas bijective; en effet, les 4 fonctions thêtas $\theta_1,...,\theta_4$ sont paires donc $\phi$ envoie deux points opposés du tore sur le même point dans $\mathcal{K}$.

On peut montrer que $\mathcal{K}$ est isomorphe au tore complexe modulo $\{\pm 1\}$. Autrement dit, on peut représenter un point de la surface de Kummer $\mathcal{K}$ sous la forme $\pm P$; où $P$ est un point du tore $\mathbb{C}^2/(\mathbb{Z}^2+\Omega\mathbb{Z}^2)$.

Cette observation nous permet de voir que l'on n'a pas tout à fait une structure de groupe sur $\mathcal{K}$; cependant, on a une structure qui s'en approche le plus possible que l'on appellera pseudo-addition. En effet, si l'on dispose de deux points $\pm P$ et $\pm Q$ sur la surface de Kummer alors on ne peut pas faire la différence entre $\pm(P+Q)$ et $\pm(P-Q)$. Cependant, si l'on dispose de $\pm P,\pm Q$ et $\pm(P-Q)$ alors on peut calculer $\pm(P+Q)$.

\subsection{Pseudo-addition et doublement}

\begin{algorithm}
\label{DOUBLE}
\SetKwInOut{Input}{Entrée}\SetKwInOut{Output}{Sortie}
\caption{Algorithme de doublement}
\Input{ Un point $P=(x,y,z,t) \in \mathcal{K}$.}
\Output{ Le point $2P$. }
$x' = (x^2 + y^2 + z^2 + t^2)^2$ \\
$y' = y'_0(x^2 + y^2 - z^2 - t^2)^2$ \\
$z' = z'_0(x^2 - y^2 + z^2 - t^2)^2$ \\
$t' = t'_0(x^2 - y^2 - z^2 + t^2)^2$ \\
$X = (x' + y' + z' + t')$ \\
$Y = y_0(x' + y' - z' - t')$ \\
$Z = z_0(x' - y' + z' - t')$ \\
$T = t_0(x' - y' - z' + t')$ \\
\Return{$(X,Y,Z,T)$}
\end{algorithm}

L'avantage de l'utilisation des fonctions thêtas pour paramétrer la surface de Kummer $\mathcal{K}$ est que l'on obtient ainsi des formules de pseudo-addition et de doublement compacte et très rapide. En particulier, la multiplication scalaire de genre 2 que l'on obtient est la plus rapide à l'heure actuelle.

Les algorithmes \ref{ADD} et \ref{DOUBLE} utilisent des constantes $y_0,z_0,t_0,y'_0,z'_0,t'_0$ qui s'expriment en fonction des thêtas constantes \citep{gaudry}.

Le calcul du doublement nécessite 6 produits et 8 élévations au carré. On voit que le coût du doublement est divisé par 2 ou 3 par rapport aux formules \ref{propDuquesne} de Dusquesne.

\begin{algorithm}
\label{ADD}
\SetKwInOut{Input}{Entrée}\SetKwInOut{Output}{Sortie}
\caption{Algorithme de pseudo-addition}
\Input{ Trois points $P=(x,y,z,t),Q=(\underline{x},\underline{y},\underline{z},\underline{t}),R=(\bar{x},\bar{y},\bar{z},\bar{t}) \in \mathcal{K}$. }
\Output{ Le point $P+Q$. }
\textbf{Hypothèse} : Le point $R$ est égal à $P-Q$ et ses coordonnées sont non nulles. \\
$x' = (x^2 + y^2 + z^2 + t^2)(\underline{x}^2 + \underline{y}^2 + \underline{z}^2 + \underline{t}^2)$ \\
$y' = y'_0(x^2 + y^2 - z^2 - t^2)(\underline{x}^2 + \underline{y}^2 - \underline{z}^2 - \underline{t}^2)$ \\
$z' = z'_0(x^2 - y^2 + z^2 - t^2)(\underline{x}^2 - \underline{y}^2 + \underline{z}^2 - \underline{t}^2)$ \\
$t' = t'_0(x^2 - y^2 - z^2 + t^2)(\underline{x}^2 - \underline{y}^2 - \underline{z}^2 + \underline{t}^2)$ \\
$X = (x' + y' + z' + t')/\bar{x}$ \\
$Y = (x' + y' - z' - t')/\bar{y}$ \\
$Z = (x' - y' + z' - t')/\bar{z}$ \\
$T = (x' - y' - z' + t')/\bar{t}$ \\
\Return{$(X,Y,Z,T)$}
\end{algorithm}

Le calcul d'une pseudo-addition nécessite 7 produits, 8 élévations au carré et 4 divisions. Comme les coordonnées sont projectives, on peut remplacer les 4 divisions par 10 multiplications; ce qui ramène à 17 produits et 8 élévations au carré le calcul d'une pseudo-addition. Le gain par rapport aux formules \ref{propDuquesne} de Duquesne est légèrement moins bon que pour le doublement, mais on divise encore pratiquement par 2 le coût d'une pseudo-addition.

\begin{remarque}
Dans l'algorithme \ref{ADD}, on suppose que les coordonnées du point $R$ sont non nulles. C'est cependant une restriction mineure étant donné que le point $R$ est toujours le même dans l'algorithme de Montgomery \ref{algoMontgomery}; on peut donc le détecter au départ et agir en conséquence.
\end{remarque}

\subsection{Multiplication scalaire}

Comme on l'a remarqué précédemment, ces formules apportent un gain significatif par rapport aux formules \ref{propDuquesne} de Duquesne; de plus, la symétrie des formules nous permet d'économiser encore plus de calcul dans l'algorithme de Montgomery car certaines opérations sont communes au doublement et à la pseudo-addition.

On remarque par exemple que les calculs de $x^2,y^2,z^2,t^2$ sont nécessaire dans la pseudo-addition et dans le doublement; on peut donc faire les calculs de la pseudo-addition et du doublement en même temps et économiser certaines opérations que l'on effectue en double. On renvoie à \citep{gaudry} pour le détail des optimisations possibles; ce qui nous permet d'obtenir le théorème suivant :
\begin{theoreme}[\citet{gaudry}]
\label{mulscal}
Soit $P$ un point de $\mathcal{K}$ dont les coordonnées sont non nulles et $n > 1$ un entier. Le calcul de $nP$ nécessite $16\log_2 n$ produits et $9\log_2 n$ élévations aux carrés.
\end{theoreme}

Ce résultat est à comparer aux approches précédentes n'utilisant pas les fonctions thêtas, Duquesne \ref{thduquesne} a besoin de $62\log_2 n$ produits et $7\log_2 n$ élévations aux carrés. Tandis que si l'on n'utilise pas les surfaces de Kummer, on a besoin de $78\log_2 n$ produits et $10\log_2 n$ élévations aux carrés pour une multiplication scalaire sur la Jacobienne $\mathcal{J}$.

\begin{remarque}
Une rapide comparaison avec les courbes elliptiques montre que la multiplication scalaire de genre 2 est la plus rapide à niveau de sécurité fixé \citep[5.5]{gaudry}; donc l'utilisation de courbe de genre 2 peut être avantageuse pour une utilisation cryptographique, comme (par exemple) une implémentation de Diffie-Hellman en genre 2.
\end{remarque}

\subsection{Points de 2-torsion}

La surface de Kummer $\mathcal{K}$ admet 16 points singuliers, on les appellera les \emph{noeuds} de la surface. Ce sont exactement les 16 points de 2-torsion de $\mathcal{K}$ :
\begin{align*}
   &(a,b,c,d),(a,b,-c,-d),(a,-b,c,-d),(a,-b,-c,d),& \\
   &(b,a,d,c),(b,a,-d,-c),(b,-a,d,-c),(b,-a,-d,c),& \\
   &(c,d,a,b),(c,d,-a,-b),(c,-d,a,-b),(c,-d,-a,b),& \\
   &(d,c,b,a),(d,c,-b,-a),(d,-c,b,-a),(d,-c,-b,a).&
\end{align*}

La 2-torsion joue un rôle important dans le calcul des formes biquadratiques et les symétries apparentes expliquent (du moins en partie) la simplicité desdites formes. 
En effet, l'approche de \citet{cassels-Flynn} pour calculer les formes biquadratiques consiste à partir du calcul des translations par un point de 2-torsion et à trouver un polynôme qui se spécialise sur la 2-torsion en l'expression correspondante.
On peut donc imaginer que plus les translations par la 2-torsion sont simples et plus les formes biquadratiques le seront aussi.

\citet{gaudry} explique que la translation par un point de 2-torsion consiste à effectuer la même permutation et les mêmes changements de signes qui expriment ledit point en partant du point $(a,b,c,d)$ qui est l'élément neutre de $\mathcal{K}$. Par exemple, le point $(y,-x,-t,z)$ correspond à la translation du point $(x,y,z,t)$ par le point $(b,-a,-d,c)$.

On voit bien que les translations par des points de 2-torsion sont plus simples (en général) que celles de \citet{eqKum}; en particulier, la translation par l'un des 4 points de la première ligne est diagonale à coefficients $\pm 1$.

On remarquera que les points de 2-torsion sont obtenus à partir du point $(a,b,c,d)$ en effectuant une double transposition (éventuellement triviale) "verticalement" et des changements de signes (éventuellement triviaux) "horizontalement".

Chacun de ces noeuds est incident à 6 plans "tangents" (dégénérés) particuliers tels que chacun de ces plans passe par exactement 6 noeuds, on appellera ces plans des \emph{tropes}. Les relations d'incidence tropes/noeuds forment une structure que l'on appelle \emph{configuration de Kummer} $16_6$; on se reportera à \citep{hudson} pour une étude plus approfondie.

Il sera commode d'utiliser une notion de dualité en géométrie projective; on identifiera dans $\mathbb{P}^3$ un plan d'équation $$\alpha X + \beta Y + \gamma Z + \delta T = 0$$ avec un point de coordonnées homogènes $(\alpha, \beta, \gamma, \delta)$ et que l'on appellera \emph{coordonnées duales} du plan. La relation d'incidence entre deux éléments $(\alpha, \beta, \gamma, \delta)$ et $(\alpha', \beta', \gamma', \delta')$ de type point/plan ou plan/point s'écrit
$$\alpha\alpha' + \beta\beta' + \gamma\gamma' + \delta\delta' = 0$$

On remarque donc que toute relation d'incidence entre un point et un plan donne une seconde relation d'incidence en échangeant les types point/plan et vice versa. On en déduit donc que toute relation d'incidence peut s'interpréter indifféremment comme une relation point/plan ou une relation plan/point et ceci en utilisant exactement les mêmes coordonnées; cependant, ces coordonnées sont interprétées différemment.

Pour trouver ces relations d'incidence tropes/noeuds, on va utiliser un analogue du fait bien connu que le plan $\alpha X + \beta Y=0$ de coordonnées duales $(\alpha,\beta)$ est incident au point $(-\beta, \alpha)$; on remarque que l'on a effectué une transposition des coordonnées et un changement de signe.

On généralise maintenant ceci pour un hyperplan dans $\mathbb{P}^3$ de coordonnées duales $(\alpha, \beta, \gamma, \delta)$; en utilisant des doubles transpositions et des changements de signes, on trouve que le plan précédent est incident aux 6 points dont les coordonnées sont :
\begin{align*}
   &(\delta, -\gamma, \beta, -\alpha), \quad (\delta, \gamma, -\beta, -\alpha), \\
   &(\gamma, \delta, -\alpha, -\beta), \quad (-\gamma, \delta, \alpha, -\beta), \\
   &(-\beta, \alpha, \delta, -\gamma), \quad (\beta, -\alpha, \delta, -\gamma).
\end{align*}

Par l'argument de dualité expliqué précédemment, le point $(\alpha, \beta, \gamma, \delta)$ est incident aux 6 plans donnés par les mêmes coordonnées (vu comme coordonnées duales).
On désignera ces 6 coordonnées ou les transformations correspondantes par le terme \emph{relations d'incidence} ou \emph{transformations d'incidence}.

On remarquera que les transformations qui apparaissent étaient déjà présentes dans les transformations associées aux points de 2-torsion c.-à-d. des doubles transpositions et des changements de signes. De plus, on remarque que les transformations d'incidence forment un système générateur du groupe des transformations associées à la 2-torsion.

On en déduit que si l'on prend comme coordonnées duales pour le plan les coordonnées d'un point de 2-torsion, alors les 6 points que l'on obtient dans les relations d'incidence sont des points de 2-torsion; donc chaque point de 2-torsion est aussi un trope et vice versa par dualité.

On peut résumer les relations d'incidence dans le tableau suivant :
\begin{center}
\begin{tabular}{ c c c c }
   $(a,b,c,d)$   & $(d,c,-b,-a)$ & $(b,-a,d,-c)$ & $(c,-d,-a,b)$ \\
   $(c,d,-a,-b)$ & $(b,a,d,c)$ & $(d,-c,-b,a)$ & $(a,-b,c,-d)$ \\
   $(d,-c,b,-a)$ & $(a,-b,-c,d)$ & $(c,d,a,b)$ & $(b,a,-d,-c)$ \\
   $(b,-a,-d,c)$ &  $(c,-d,a,-b)$ & $(a,b,-c,-d)$ & $(d, c, b, a)$
 \end{tabular}
\end{center}

Les relations d'incidences associées à un élément du tableau correspondent aux éléments de la même ligne ou de la même colonne en enlevant l'élément dont on veut connaitre lesdites relations.

Le tableau précédent se construit en plaçant arbitrairement la diagonale puis en utilisant des corrélations entre les relations d'incidence. Par exemple, en supposant la diagonale fixée, l'élément du tableau en bas à gauche est alors l'unique élément qui est présent à la fois dans les relations d'incidence de l'élément en haut à gauche du tableau et dans celles de l'élément en bas à droite du tableau.

\subsection{Surface de Kummer autoduale}

En suivant les notations de Gaudry, la duale de la surface de Kummer $K_{a,b,c,d}$ est $K_{A,B,C,D}$; où les paramètres sont reliées par le système d'équations suivant \citep{gaudry} :
\begin{align}
\begin{split}
\label{constDual}
4A^2 &= a^2 + b^2 + c^2 + d^2 \\
4B^2 &= a^2 + b^2 - c^2 - d^2 \\
4C^2 &= a^2 - b^2 + c^2 - d^2 \\
4D^2 &= a^2 - b^2 - c^2 + d^2
\end{split}
\end{align}

On cherche des conditions pour que ces surfaces de Kummer soient isomorphes entre elles; on dira alors que la surface de Kummer est \emph{autoduale}. 

Les coefficients de la surface de Kummer $K_{a,b,c,d}$ sont homogènes de degré 0 par rapport aux variables $a,b,c,d$; il nous suffit donc de trouver des conditions pour que projectivement l'on ait $(A:B:C:D)=(a:b:c:d)$. Cette dernière condition n'est pas forcément nécessaire; cependant, on s'intéresse ici uniquement à trouver une sous-famille de surfaces de Kummer autoduale.

Si les constantes $a,b,c,d$ et $A,B,C,D,$ sont projectivement équivalentes, alors l'isogénie $\phi$ qui apparait dans le doublement de Gaudry est un endomorphisme de $\mathcal{K}$. Cette isogénie est alors une racine carrée de 2 c.-à-d. vérifie $\phi^2 = [2]$ où l'on note $[2]$ pour la multiplication par 2; dans ce cas, on notera alors $\phi$ par le symbole $\sqrt{2}$.

Revenons maintenant au système d'équations et remplaçons $A$, $B$, $C$, $D$ par $\lambda a$, $\lambda b$, $\lambda c$, $\lambda d$ dans le système \ref{constDual}; on a alors le système suivant :
\begin{align*}
4\lambda^2a^2 &= a^2 + b^2 + c^2 + d^2 \\
4\lambda^2b^2 &= a^2 + b^2 - c^2 - d^2 \\
4\lambda^2c^2 &= a^2 - b^2 + c^2 - d^2 \\
4\lambda^2d^2 &= a^2 - b^2 - c^2 + d^2
\end{align*}

On obtient un système linéaire en $a^2,b^2,c^2,d^2$; en supposant que l'on peut extraire des racines carrées, le système précédent devient :
\begin{align*}
\mu x &= x + y +  z + t \\
\mu y &= x + y -  z - t \\
\mu z &= x - y +  z - t \\
\mu t &= x - y -  z + t
\end{align*}

Les valeurs propres du système sont $\mu = 2$, auquel cas le système se réduit à l'unique équation $\alpha = \beta + \gamma + \delta$; et $\mu = -2$, l'espace propre associé est alors engendré par le vecteur $(1,-1,-1,-1)$.

En revenant aux coordonnées originales, on trouve alors comme solution $\lambda = \frac{1}{\sqrt 2}$, dans ce cas l'équation devient $a^2 = b^2 + c^2 + d^2$; ou alors $\lambda = \frac{1}{\sqrt{-2}}$ et les solutions sont engendrées par les vecteurs $(1,\sqrt{-1},\sqrt{-1},\sqrt{-1})$.

Un quadruplet $(a,b,c,d)$ vérifiant $a^2 = b^2 + c^2 + d^2$ est appelé \emph{quadruplet Pythagoricien}. \citet{mordell} donne la paramétrisation suivante :
\begin{align*}
a &= m^2 + n^2 + p^2 + q^2 \\
b &= m^2 + n^2 - p^2 - q^2 \\
c &= 2(mq+np) \\
d &= 2(nq-mp)
\end{align*}
où $m,n,p,q$ sont des entiers naturels. En effet, on vérifiera aisément\footnote{Professor Littlewood, when he makes use of an algebraic identity, always saves himself the trouble of proving it;
he maintains that an identity, if true, can be verified in a few
lines by anybody obtuse enough to feel the need of verification. -- \citet{dyson}} que l'on a bien l'identité suivante :
$$(m^2 + n^2 + p^2 + q^2)^2 = (m^2 + n^2 - p^2 - q^2)^2 + 4(mq+np)^2 + 4(nq-mp)^2$$

\begin{remarque}
On cherche à utiliser ces formules dans un corps fini $\mathbb{F}_p$, il semble donc nécessaire que le facteur projectif $\lambda$ soit définit sur $\mathbb{F}_p$. Cependant, dans $\mathbb{F}_{p^2}$, les points $(A:B:C:D)$ et $(1:B/A:C/A:D/A)$ sont égaux; de plus, le point $(1:B/A:C/A:D/A)$ est bien définit sur $\mathbb{F}_p$.

Ceci nous assure que la surface de Kummer $K_{1,B/A,C/A,D/A}$ est définit sur $\mathbb{F}_p$ et est bien la duale de $K_{a,b,c,d}$; on arrive donc à contourner le problème si jamais $2$ ou $-2$ ne sont pas des résidus quadratiques sur $\mathbb{F}_p$.
\end{remarque}

%faut donc dans le premier cas que 2 soit un résidu quadratique modulo $p$ ce qui impose la condition $p \equiv \pm 1 \mod{8}$. Dans le second cas, il faut que $-2$ et $-1$ soit des résidus quadratiques modulo $p$ ce qui impose que $2$ le soit aussi; ce qui donne la condition $p \equiv 1 \mod{8}$.

Il peut sembler étrange que dans la relation $a^2 = b^2 + c^2 + d^2$, le paramètre $a$ ne joue pas le même rôle que $b$, $c$ ou $d$. Que se passe-t-il si l'on a plutôt la relation $b^2 = a^2 + c^2 + d^2$. On obtient alors les relations suivantes qui relient les thêtas constantes aux thêtas constantes duales :
\begin{align*}
4A^2 &= 2b^2  &4B^2 &= 2a^2 \\
4C^2 &= -2d^2  &4D^2 &= -2c^2
\end{align*}
Alors qu'en utilisant la relation $a^2 = b^2 + c^2 + d^2$, on obtenait :
\begin{align*}
4A^2 &= 2a^2  &4B^2 &= 2b^2 \\
4C^2 &= 2c^2  &4D^2 &= 2d^2
\end{align*}

On voit donc que l'on a échangé les rôles respectifs de $a,b,c,d$ ce qui explique que l'on ne pouvait pas obtenir d'autres relations en cherchant uniquement une colinéarité entre $(a,b,c,d)$ et $(A,B,C,D)$.

Maintenant que l'on a vu ce qui se passe si lorsque l'on échange les rôles de $a$, $b$, $c$, $d$ dans la relation $a^2 = b^2 + c^2 + d^2$; on peut se demander ce qu'il se passe lorsque les points $(a,b,c,d)$ et $(A,B,C,D)$ ne sont pas projectivement équivalent, mais en supposant tout de même que la surface de Kummer soit autoduale.

On remarque alors que les deux isogénies qui apparaissent dans le doublement font intervenir pour la première les constantes $A/B,A/C,A/D$ et pour la seconde les constantes $a/b,a/c,a/d$. Ces constantes sont différentes car les points ne sont pas projectivement équivalents; on en déduit que les deux isogénies qui apparaissent dans le doublement sont distinctes.

Les surfaces de Kummer $\mathcal{K}$ et $\hat{\mathcal{K}}$ sont alors isomorphes mais n'ont pas la même équation. Le calcul de $\sqrt{2}$ fait alors intervenir un isomorphisme non trivial entre $\mathcal{K}$ et $\hat{\mathcal{K}}$; ceci nuit grandement au calcul rapide de $\sqrt{2}$ et ne nous intéressera pas pour des applications cryptographiques.

%\emph{Y aurait-il un argument théorique qui permette d'affirmer que lorsque la surface de Kummer est autoduale alors les deux isogénies sont égales ? Et ainsi, éliminer le cas où $(a,b,c,d)$ et $(A,B,C,D)$ ne sont pas projectivement équivalent.}

%Lors de notre analyse pour obtenir une surface de Kummer autoduale, on a vu que l'on avait à priori besoin d'extraire une racine carrée de 2. On peut alors se poser la question de savoir si on a effectivement besoin de cette racine carrée de 2 dans le corps de base. En effet, l'équation de la Kummer $K_{A,B,C,D}$ ne fait intervenir que les carrés des paramètres et le produit $ABCD$; la racine carrée de 2 n'est donc pas nécessaire dans l'équation de la Kummer.
%
%D'autre part, dans l'isogénie on n'a besoin que des quotients $A/B$, $A/C$, $A/D$ dans lesquels la racine carrée de 2 disparait miraculeusement. Cependant, les relations entre les paramètres duaux sont de la forme $2A^2 = a^2$. Si l'on veut que les points $(a,b,c,d)$ et $(A,B,C,D)$ soit rationnels; ce qui semble indispensable, car ce sont les neutres de $K_{a,b,c,d}$ et $K_{a,b,c,d}$. Alors on obtient $2 = (\frac{a}{A})^2$, ce qui nous assure que 2 doit être un carré.

\begin{remarque}
Dans la pseudo addition, on voit apparaitre uniquement l'isogénie duale. Ce qui peut sembler étrange puisqu'il apparait alors une certaine dissymétrie contrairement au doublement.
\end{remarque}

\section{Vers une arithmétique à la Gaudry}

Dans cette section, on s'intéresse à des conditions suffisantes pour qu'une courbe de genre 2 admette une arithmétique rapide; en particulier, l'existence d'une isogénie de Richelot jouera un rôle très important. Comme nous l'avons expliqué précédemment, pour avoir une arithmétique rapide, il est important que les translations par un point de 2-torsion soient les plus simples possible; c'est pourquoi on s'efforcera de diagonaliser le plus possible ces translations.

\subsection{Isogénie de Richelot}

On remarque que l'algorithme \ref{DOUBLE} fait apparaître un découpage en deux parties similaires qui correspondent à une \emph{isogénie de Richelot}. La première isogénie envoie un point de $\mathcal{K}$ sur la surface de Kummer duale $\hat{\mathcal{K}}$ et la deuxième renvoie le point sur $\mathcal{K}$. On s'intéressera plus tard au cas où les deux isogénies sont identiques; c'est à dire lorsque $\mathcal{K}$ est isomorphe à sa duale $\hat{\mathcal{K}}$.

On suppose que la courbe $C$ de genre 2 admet une équation de la forme :
$$y^2 = F(x) = G_1(x)G_2(x)G_3(x)$$
où $G_1(X),G_2(X),G_3(X)$ sont des polynômes de degré 2 sans racine double et sans racine commune. De plus, on note :
$$G_j(X) = g_{2j}X^2 + g_{1j}X + g_{0j}$$

\begin{remarque}
Une telle factorisation du polynôme $F(X)$ permet de distinguer 3 points de 2-torsion particuliers (en plus de l'élément neutre) qui correspondent à $G_1(X)$, $G_2(X)$, $G_3(X)$ et qui joueront un rôle important dans la suite.
\end{remarque}

\begin{definition}
Une courbe $\hat{C}$ est dite \emph{duale de Richelot} à la courbe $C$ si elle admet une équation de la forme :
$$\hat{C} : \Delta y^2 = L_1(x)L_2(x)L_3(x)$$
où $\Delta = \det(g_{ij})$ et les polynômes $L_j(X)$ sont définies par :
\begin{align*}
L_1(X) &= G'_2(X)G_3(X) - G'_3(X)G_2(X) \\
L_2(X) &= G'_3(X)G_1(X) - G'_1(X)G_3(X) \\
L_3(X) &= G'_1(X)G_2(X) - G'_2(X)G_1(X)
\end{align*}
\end{definition}

\begin{remarque}
On fera attention au fait qu'il y a potentiellement plusieurs duales de Richelot $\hat{C}$ pour une même courbe $C$, correspondant à plusieurs factorisations du polynôme $F(X)$; il nous arrivera cependant de parler de \emph{la} duale de Richelot lorsque la factorisation est fixée.
\end{remarque}

\begin{remarque}
Il peut arriver que l'on ait $\Delta = 0$, auquel cas la Jacobienne $\mathcal{J}$ est un produit de courbes elliptiques \citep{cassels-Flynn}. Dans la suite, on exclut ce cas qui ne nous est pas favorable pour des applications cryptographiques puisque le logarithme discret sur $\mathcal{J}$ se ramène alors à deux calculs de logarithme discret dans des groupes plus petits.
\end{remarque}

Un calcul montre que si l'on applique le même procédé à $L_1,L_2,L_3$ on retrouve les polynômes $G_1,G_2,G_3$ à une constante prés. De plus l'analogue de $\Delta$ pour $\hat{C}$ est $2\Delta^2$; on trouve alors une courbe isomorphe à $C$. Autrement dit, la biduale $\hat{\hat{C}}$ est isomorphe à la courbe $C$ de départ.

\begin{definition}
La surface de Kummer duale $\hat{\mathcal{K}}$ est la surface de Kummer associée à la duale de Richelot $\hat{C}$.
\end{definition}

Comme nous l'avons déjà remarqué, la décomposition du polynôme $F(X)$ en produit de 3 polynômes $G_j(X)$ fait apparaitre 4 points de 2-torsion particuliers; le neutre $(0,0,0,1)$ et les diviseurs donnés par $G_j=0$.

\begin{definition}
On appelle \emph{tétraèdre de G\"opel} un ensemble de 4 points de $\mathcal{K}$ formé de l'élément neutre $(0,0,0,1)$ et de 3 points donnés par $G_j=0$; où $G_1(X)G_2(X)G_3(X)$ est une factorisation du polynôme $F(X)$ et où les polynômes $G_j$ sont de degré $2$.
\end{definition}

Un tétraèdre de G\"opel est particulièrement adapté à l'isogénie de Richelot correspondante et l'on va voir que les translations dans cette base sont relativement simples; on pourra comparer les applications linéaires avec celles que l'on a vues dans la section \ref{add2tors}.

Une propriété importante des applications linéaires associées au tétraèdre de G\"opel est qu'elles commutent entre elles \citep{cassels-Flynn}; ce qui nous permettra de les diagonaliser simultanément. 

Plus précisément, on a la proposition suivante :

\begin{proposition}
Notons $S_j$ la matrice de translation par le point de 2-torsion donnée par $G_j=0$; alors il existe des éléments non nuls $e_1,e_2,e_3$ de $k$ tel que l'on ait les relations suivantes :
\begin{align*}
S_3S_2 &= S_2S_3 = e_1S_1 \\
S_1S_3 &= S_3S_1 = e_2S_2 \\
S_2S_1 &= S_1S_2 = e_3S_3
\end{align*}
De plus, en combinant les lignes $j+1$ et $j+2$, on obtient :
$$S_j^2 = e_{j+1}e_{j+2}I$$
\end{proposition}

On note $O=(0,0,0,1)$ l'élément neutre.
\begin{remarque}
Si les 4 points $O$, $S_1O$, $S_2O$, $S_3O$ sont linéairement dépendants, alors la Jacobienne $\mathcal{J}$ est réductible \citep{cassels-Flynn}; c.-à-d. produit de deux courbes elliptiques. On exclut ce cas et on suppose dans la suite que $(O,S_1O,S_2O,S_3O)$ est une base de $\mathbb{P}^3$.
\end{remarque}

La base $(O,S_1O,S_2O,S_3O)$ est particulièrement adaptée à l'isogénie de Richelot et les translations par un point du tétraèdre de G\"opel sont relativement simples dans cette base. Par exemple, la matrice de $S_1$ dans la base  $(O,S_1O,S_2O,S_3O)$ est :
\[ \left(
  \begin{array}{ c c c c}
     0 & e_2e_3 & 0  & 0  \\
     1 & 0   & 0  & 0  \\
     0 & 0   & 0  & e_2 \\
     0 & 0   & e_3  & 0  \\
  \end{array} \right)
\]
De plus, le polynôme caractéristique de cette matrice est :
$$(X^2-e_2e_3)^2$$

Si l'on veut donc diagonaliser complètement les translations par un point du tétraèdre de G\"opel (comme c'est le cas dans \citep{gaudry}), il nous faut extraire certaines racines carrées.

Notons $\epsilon_j$ une racine carrée de $e_j$ et considérons le changement de base suivant \citep{cassels-Flynn} :
\begin{align}
\begin{split}
\label{chgBase}
X &= \epsilon_1\epsilon_2\epsilon_3(v_0+v_1+v_2+v_3) \\
Y &= \epsilon_1(v_0+v_1-v_2-v_3) \\
Z &= \epsilon_2(v_0-v_1+v_2-v_3) \\
T &= \epsilon_3(v_0-v_1-v_2+v_3)
\end{split}
\end{align}
où $X,Y,Z,T$ sont les coordonnées dans la base du tétraèdre de G\"opel et $v_0,v_1,v_2,v_3$ sont les nouvelles coordonnées.

Dans la nouvelle base, les translations par un point du tétraèdre de G\"opel sont diagonales à coefficients $\pm 1$ (projectivement); par exemple, la matrice de $S_1$ dans cette base est :
\[ \epsilon_1\epsilon_2
\left(
  \begin{array}{ c c c c}
     1 & 0 & 0  & 0  \\
     0 & -1   & 0  & 0  \\
     0 & 0   & -1  & 0 \\
     0 & 0   & 0  & 1  \\
  \end{array} \right)
\]

\begin{remarque}
Si l'on divise les équations par $\epsilon_1\epsilon_2\epsilon_3$, les seules constantes qui apparaissent sont $\epsilon_1\epsilon_2$,$\epsilon_1\epsilon_3$ et $\epsilon_2\epsilon_3$. Ce qui nous permet d'obtenir le corollaire suivant :
\end{remarque}

\begin{corollaire}
Pour diagonaliser les translations par un point du tétraèdre de G\"opel, il suffit que les éléments $e_1e_2$, $e_1e_3$ et $e_2e_3$ soient des carrés dans le corps de base.
\end{corollaire}

On fera attention lorsque l'on extrait des racines carrées de $e_1e_2$, $e_1e_3$ et $e_2e_3$ de faire un choix consistant; en effet, si l'on pose :
$$\epsilon_{12} = \sqrt{e_1e_2}, \quad \epsilon_{13} = \sqrt{e_1e_3}, \quad \epsilon_{23} = \sqrt{e_2e_3}$$
Il nous faut vérifier que l'on a bien l'égalité $\epsilon_{12}\epsilon_{13} = e_1\epsilon_{23}$.
Pour ce faire, on peut de calculer $\epsilon_{12}$ et $\epsilon_{13}$ en prenant une racine carrée arbitraire puis poser $\epsilon_{23} = \epsilon_{12}\epsilon_{13}/e_1$.

\begin{remarque}
La base correspondante n'est pas encore la base de \citep{gaudry} mais on s'y intéressera plus en détail dans la suite, pour des questions de généricité et de rationalité. En effet, on arrive à diagonaliser une partie de la 2-torsion et on peut imaginer que les formes biquadratiques seront suffisamment simples pour être intéressantes tout s'autorisant une classe de courbes plus large que dans \citep{gaudry}.
\end{remarque}

On reprend dans ce qui suit l'explication de Cassels et Flynn en ce qui concerne l'expression de l'isogénie de Richelot sur la surface de Kummer; ils ne donnent pas l'expression exacte mais on va tout de même essayer d'extrapoler légèrement pour obtenir quelque chose qui devrait ressembler à l'expression de l'isogénie.

L'équation de la surface de Kummer $\mathcal{K}$ dans les variables $v_i$ est de la forme suivante \citep{cassels-Flynn} :
\begin{equation}
\label{eqKumDiag}
\mathcal{K} : q(v_0^2,v_1^2,v_2^2,v_3^2)+\gamma v_0v_1v_2v_3 = 0
\end{equation}
où $q$ est une forme quadratique et $\gamma$ une constante.

Si l'on pose $V_i = v_i^2$, les variables $V_i$ vérifient alors
\begin{equation}
\label{dualeKum}
q(V_0,V_1,V_2,V_3)^2 = \gamma^2 V_0V_1V_2V_3
\end{equation}

L'équation \ref{dualeKum} est l'équation de la surface de Kummer duale $\hat{\mathcal{K}}$ \citep{cassels-Flynn}; on cherche maintenant à mettre l'équation \ref{dualeKum} sous la forme \ref{eqKumDiag}. On utilise pour ce faire un changement de base similaire à \ref{chgBase}; l'isogénie de Richelot devrait avoir une expression de la forme
\begin{align*}
X &= \alpha_0v_0^2+\alpha_1v_1^2+\alpha_2v_2^2+\alpha_3v_3^2 \\
Y &= \alpha_0v_0^2+\alpha_1v_1^2-\alpha_2v_2^2-\alpha_3v_3^2 \\
Z &= \alpha_0v_0^2-\alpha_1v_1^2+\alpha_2v_2^2-\alpha_3v_3^2 \\
T &= \alpha_0v_0^2-\alpha_1v_1^2-\alpha_2v_2^2+\alpha_3v_3^2
\end{align*}
où les constantes $\alpha_0,\alpha_1,\alpha_2,\alpha_3$ seront à déterminer.

\begin{remarque}
Comme on va le voir dans la suite, l'isogénie ne s'exprime pas tout à fait sous cette forme; il manque les constantes duales qui vont apparaître. Cependant, on n'arrive pas à trouver de bonne explication pour l'apparition de ces constantes.
\end{remarque}

On remarquera la similarité de ces expressions avec les formules de l'algorithme \ref{DOUBLE}; on a aussi une isogénie de Richelot sur $\hat{\mathcal{K}}$ qui correspond à la deuxième partie de l'algorithme \ref{DOUBLE}.

\begin{remarque}
Comme nous l'avons déjà fait remarquer, les courbes de Montgomery disposent d'une 2-isogénie; ce qui est (selon toute vraisemblance) une composante essentielle pour avoir une arithmétique rapide.
L'existence d'une isogénie de Richelot simple est donc un bon point de départ pour un analogue de genre 2 des courbes de Montgomery.
\end{remarque}
\subsection{Calcul des coefficients}

En partant de la paramétrisation par les fonctions thêtas de \citep{gaudry} et en effectuant le changement de base
$$x'=ax, \quad y'=by, \quad z'=cz, \quad t'=dt;$$
on se rend compte que les constantes qui apparaissent dans les biquadratiques (les constantes $\alpha_0,\alpha_1,\alpha_2,\alpha_3$ que l'on cherche à déterminer) sont essentiellement les thêtas constantes au carré; c.-à-d. $a^2,b^2,c^2,d^2$. On peut alors les calculer à partir de certains coefficients de l'équation de la surface de Kummer $\mathcal{K}$.

Pour des raisons de simplicité, on va supposer que le polynôme $G_1$ est de la forme $X^2 + g_1$ (ce que l'on peut supposer en effectuant une translation) et l'on note :
\begin{align*}
G_1 &= X^2 + g_1 \\
G_2 &= X^2+g_2X+g_3 \\
G_3 &= X^2+g_4X+g_5
\end{align*}

On suppose que l'équation de la surface de Kummer sous la forme \ref{eqKumDiag} est :
$$K(\xi_1,\xi_2,\xi_3,\xi_4) = c_1\xi^4_1 + c_2\xi^4_2 + c_3\xi^4_3 + c_4\xi^4_4 + ...$$ 

En effectuant un \href{run:./chgBase.sag}{changement de base} à partir de l'équation de \citep{eqKum}, on obtient les relations suivantes qui nous permettent de calculer ces coefficients :
\begin{align*}
c_1+c_2+c_3+c_4 &= \lambda(-2g_1Res(G_2,G_3)-2g_3Res(G_1,G_3)-2g_5Res(G_2,G_1) \\
 &+ g_1g_2^2g_4^2 + g_1^2g_2g_4 - g_1g_2g_3g_4 - g_1g_2g_4g_5 + g_2g_3g_4g_5) \\
c_1+c_2-c_3-c_4 &= \lambda(-g_2g_4+2g_3+2g_5)\epsilon_2\epsilon_3 \\
c_1-c_2+c_3-c_4 &= 2\lambda(g_1+g_5)\epsilon_1\epsilon_3 \\
c_1-c_2-c_3+c_4 &= 2\lambda(g_1+g_3)\epsilon_1\epsilon_2
\end{align*}
où
$$\lambda = 32Res(G_1,G_2)Res(G_2,G_3)Res(G_1,G_3)$$

Une fois que l'on a calculé ces coefficients et en revenant aux thêtas constantes, on sait que l'on a les égalités suivantes :
$$a^4 = c_1, \quad b^4 = c_2, \quad c^4 = c_3, \quad d^4 = c_4$$

On calcule alors une partie des constantes de l'isogénie en prenant simplement des racines carrées de $c_1,c_2,c_3,c_4$ et pour les constantes duales il nous suffit d'utiliser les relations \ref{constDual}.

On peut alors vérifier sur quelques exemples que les formules sont correctes en les comparant avec les biquadratiques qui viennent de \citep{biquad} après changement de base.

On obtient alors facilement les algorithmes \ref{DOUBLE2} et \ref{ADD2} en partant des formules de \citep{gaudry} et en effectuant un changement de variable pour revenir dans la bonne base; où les constantes dans les algorithmes \ref{DOUBLE2} et \ref{ADD2} sont reliés aux thêtas constantes par les relations :
$$y_0 = (b/a)^2, \quad z_0 = (c/a)^2, \quad t_0 = (d/a)^2$$
et
$$y'_0 = (B/A)^2, \quad z'_0 = (C/A)^2, \quad t'_0 = (D/A)^2$$

\begin{algorithm}
\label{DOUBLE2}
\SetKwInOut{Input}{Entrée}\SetKwInOut{Output}{Sortie}
\caption{Algorithme de doublement}
\Input{ Un point $P=(x,y,z,t) \in \mathcal{K}$}
\Output{ Le point $2P$ }
$x' = (x^2 + y_0y^2 + z_0z^2 + t_0t^2)^2$ \\
$y' = 1/y'_0(x^2 + y_0y^2 - z_0z^2 - t_0t^2)^2$ \\
$z' = 1/z'_0(x^2 - y_0y^2 + z_0z^2 - t_0t^2)^2$ \\
$t' = 1/t'_0(x^2 - y_0y^2 - z_0z^2 + t_0t^2)^2$ \\
$X = (x' + y' + z' + t')$ \\
$Y = 1/y_0(x' + y' - z' - t')$ \\
$Z = 1/z_0(x' - y' + z' - t')$ \\
$T = 1/t_0(x' - y' - z' + t')$ \\
\Return{$(X,Y,Z,T)$}
\end{algorithm}

L'algorithme \ref{DOUBLE2} montre un découpage du doublement optimisé pour minimiser le nombre d'opérations, un autre découpage va nous permettre de mettre en évidence l'isogénie de Richelot.

On peut, en revenant aux thêtas constantes, écrire le doublement sous la forme : 
\begin{align*}
x' &= 1/A^2(a^2x^2 + b^2y^2 + c^2z^2 + d^2t^2) \\
y' &= 1/B^2(a^2x^2 + b^2y^2 - c^2z^2 - d^2t^2) \\
z' &= 1/C^2(a^2x^2 - b^2y^2 + c^2z^2 - d^2t^2) \\
t' &= 1/D^2(a^2x^2 - b^2y^2 - c^2z^2 + d^2t^2) \\
X &= 1/a^2(A^2x'^2 + B^2y'^2 + C^2z'^2 + D^2t'^2) \\
Y &= 1/b^2(A^2x'^2 + B^2y'^2 - C^2z'^2 - D^2t'^2) \\
Z &= 1/c^2(A^2x'^2 - B^2y'^2 + C^2z'^2 - D^2t'^2) \\
T &= 1/d^2(A^2x'^2 - B^2y'^2 - C^2z'^2 + D^2t'^2) \\
\end{align*}

On a bien un découpage du doublement en deux parties qui se ressemblent où l'on a simplement échangé les constantes $a,b,c,d$ et $A,B,C,D$; ce qui nous fait penser que l'on a bien l'isogénie de Richelot et son isogénie duale. 

Il ne nous reste plus qu'à vérifier que l'on a bien une isogénie de Richelot et sa duale. Pour ce faire, on va utiliser une propriété importante des isogénies de Richelot :
\begin{proposition}
Tout tétraèdre de G\"opel détermine uniquement à isomorphisme prés une isogénie de Richelot, ce tétraèdre étant le noyau $\cong (\mathbb{Z}/2\mathbb{Z})^2$ d'une (2,2)-isogénie.
\end{proposition}

Il nous suffit alors de calculer le noyau du premier morphisme, que l'on notera $\phi$. On montre facilement que lorsque l'on cherche à résoudre le système d'équations
$$x' = y' = z' = t' = 1$$
On trouve alors que l'on a (projectivement) :
$$x^2 = y^2 = z^2 = t^2 = 1$$
Autrement dit, que le noyau de $\phi$ est bien le tétraèdre de G\"opel :
$$(1,1,1,1),(1,1,-1,-1),(1,-1,1,-1),(1,-1,-1,1)$$

%De la même façon, on montre que le noyau de $\hat{\phi}$ est identique; de plus, dans la clôture algébrique la 2-torsion est somme directe de ce noyau et du sous-groupe formé des points suivants :
%$$(1,1,1,1),(b/a,a/b,d/c,c/d),(c/a,d/b,a/c,b/d),(d/a,c/b,b/c,a/d)$$
%On montre alors que l'image de la 2-torsion par $\phi$ correspond au noyau de $\hat{\phi}$.
Ce qui montre bien que $\phi$ est une $(2,2)$-isogénie; donc $\phi$ est une isogénie de Richelot.

\begin{algorithm}
\label{ADD2}
\SetKwInOut{Input}{Entrée}\SetKwInOut{Output}{Sortie}
\caption{Algorithme de pseudo-addition}
\Input{ Trois points $P=(x,y,z,t),Q=(\underline{x},\underline{y},\underline{z},\underline{t}),R=(\bar{x},\bar{y},\bar{z},\bar{t}) \in \mathcal{K}$}
\Output{ Le point $P+Q$ }
\textbf{Hypothèse} : Le point $R$ est égal à $P-Q$ et ses coordonnées sont non nulles. \\
$x' = (x^2 + y_0y^2 + z_0z^2 + t_0t^2)(\underline{x}^2 + y_0\underline{y}^2 + z_0\underline{z}^2 + t_0\underline{t}^2)$ \\
$y' = 1/y'_0(x^2 + y_0y^2 - z_0z^2 - t_0t^2)(\underline{x}^2 + y_0\underline{y}^2 - z_0\underline{z}^2 - t_0\underline{t}^2)$ \\
$z' = 1/z'_0(x^2 - y_0y^2 + z_0z^2 - t_0t^2)(\underline{x}^2 - y_0\underline{y}^2 + z_0\underline{z}^2 - t_0\underline{t}^2)$ \\
$t' = 1/t'_0(x^2 - y_0y^2 - z_0z^2 + t_0t^2)(\underline{x}^2 - y_0\underline{y}^2 - z_0\underline{z}^2 + t_0\underline{t}^2)$ \\
$X = (x' + y' + z' + t')/\bar{x}$ \\
$Y = (x' + y' - z' - t')/(y_0\bar{y})$ \\
$Z = (x' - y' + z' - t')/(z_0\bar{z})$ \\
$T = (x' - y' - z' + t')/(t_0\bar{t})$ \\
\Return{$(X,Y,Z,T)$}
\end{algorithm}

En tenant compte du fait que les constantes $y_0$, $z_0$, $t_0$, $y'_0$, $z'_0$, $t'_0$ correspondent aux quotients $(b/a)^2$, $(c/a)^2$, $(d/a)^2$, $(B/A)^2$, $(C/A)^2$,$(D/A)^2$; on peut les calculer à partir de $c_1$, $c_2$, $c_3$, $c_4$ par les relations suivantes :
$$y_0 = \sqrt{\frac{c_2}{c_1}}, \quad z_0 = \sqrt{\frac{c_3}{c_1}}, \quad t_0 = \sqrt{\frac{c_4}{c_1}}$$
$$y'_0 = \frac{1+y_0-z_0-t_0}{1+y_0+z_0+t_0}, \quad z'_0 = \frac{1-y_0+z_0-t_0}{1+y_0+z_0+t_0}, \quad
t'_0 = \frac{1-y_0-z_0+t_0}{1+y_0+z_0+t_0}$$

\begin{remarque}
Le seul obstacle (mineur) qu'il reste à résoudre est un problème de signe qui peut apparaitre lorsque l'on prend des racines carrées; en effet, on fait un choix arbitraire de signe qui peuvent entrainer des incohérences.

Si l'on effectue un nombre pair changement de signes, ce n'est pas très grave puisque cela correspond seulement à une permutation des opérations. Par contre, si l'on effectue un nombre impair de changements de signes alors les formules sont tout simplement fausses.

Pour l'instant, nous ne savons pas comment décider si l'on a fait un choix de signe consistant, un étude plus approfondie de la thèse de \citet{cosset} pourrait nous en apprendre suffisamment pour résoudre le problème.
\end{remarque}

Dans tous les cas, à un signe près on a des formules analogues (décrites dans les algorithmes \ref{DOUBLE2} et \ref{ADD2} ) à celles de \citep{gaudry}.

\begin{remarque}
En réponse à une question de Gaudry, on remarque qu'il y a des cas où la 2-torsion n'est pas entièrement rationnelle; contrairement à ce qu'impose l'utilisation des fonctions thêtas \citep{gaudry}.
\end{remarque}

Si l'on veut maintenant passer dans la base de \citet{gaudry}, il faut supposer que les constantes $c_i$ sont des puissances 4-ième alors que jusqu'à présent on a seulement supposer que les quotients $c_2/c_1$, $c_3/c_1$ et $c_4/c_1$ sont des carrés; on effectue alors le changement de base
$$x'=x/c_1^{1/4}, \quad y'=y/c_2^{1/4}, \quad z'=z/c_3^{1/4}, \quad t'=t/c_4^{1/4}$$ pour passer dans la base de \citet{gaudry}. Il faudra (encore une fois) faire attention au choix des racines 4-ièmes.

\subsection{Relèvement vers la Jacobienne}
Une des formules présentes dans \citep{gaudry} concerne le calcul d'une représentation de Mumford associé à un point de $\mathcal{K}$. La formule est compliquée et on ne comprend pas bien d'où elle provient.

Les formules de changement de base que l'on propose permettent de simplifier (ou pas) cette application de la surface de Kummer $\mathcal{K}$ vers la Jacobienne $\mathcal{J}$. En effet, à partir d'un point de $\mathcal{K}$ dans la base de \citet{gaudry}, on effectue la transformation linéaire que l'on a trouvée précédemment pour l'avoir dans la base de \citet{cassels-Flynn}; finalement, on peut facilement remonter à une représentation de Mumford dans la Jacobienne $\mathcal{J}$.

En effet, les coordonnées sur la surface de Kummer de \citet{cassels-Flynn} sont de la forme $(1,x+u,xu,\beta_0(x,y,u,v))$; si l'on note $<a,b>$ les coordonnées de Mumford d'un relèvement dans $\mathcal{J}$, on trouve facilement le polynôme $a$ en fonction des 3 premières coordonnées de la surface de Kummer (les coefficients du polynôme $a$ sont exactement $-(x+u)$ et $xu$).

D'autre part, on a :
$$\beta_0(x,y,u,v) = \frac{F_0(x,u)-2yv}{(x-u)^2}$$

Ce qui va nous permettre de calculer $yv$ en fonction des coordonnées de $\mathcal{K}$ grâce à l'égalité suivante :
\begin{align*}
yv &= -\frac{(x-u)^2\beta_0 - F_0}{2} \\
   &= -\frac{1}{2}(\beta_0((x+u)^2-4xu) - 2f_0 - f_1(x+u) - 2f_2xu - f_3(x+u)xu \\
   & \quad - 2f_4(xu)^2 - f_5(x+u)(xu)^2 - 2f_6(xu)^3)
\end{align*}

Ce qui est le calcul du coefficient constant du polynôme $b$; pour ce qui est du dernier coefficient, on a
\begin{align*}
(y+v)^2 &= y^2 + v^2 + 2yv \\
  &= F(x) + F(u) + 2yv
\end{align*}

L'expression $F(x) + F(u)$ étant symétrique en $x$ et $u$, on peut l'exprimer en fonction de $x+u$ et $xu$; en effet, on vérifie que l'on a bien
\begin{align*}
F(x) + F(u) &= 2f_0 + f_1(x+u)+ f_2((x+u)^2 - 2xu) + f_3((x+u)^3 \\ 
   &- 3(x+u)xu) + f_4((x+u)^4 - 4(x+u)^2xu + 2(xu)^2)   \\
   &+ f_5((x+u)^5 - 5(x+u)^3xu+ 5(x+u)(xu)^2) \\
   &+ f_6((x+u)^6 - 6(x+u)^4xu + 9(x+u)^2(xu)^2 - 2 (xu)^3)
\end{align*}

Ce qui donne une expression du dernier coefficient au carré en fonction des coordonnées de la surface de Kummer.

\section{Courbe à multiplication réelle}

Dans cette section, on présente la famille de courbe de Bending qui dispose d'une multiplication par $\sqrt{2}$ et l'on montre comment utiliser cette $\sqrt{2}$ pour avoir une multiplication scalaire encore plus rapide.

\begin{definition}
On dit qu'une courbe $C$ de genre 2 est munie d'une multiplication réelle s'il existe un endomorphisme $\phi$ de la Jacobienne $\mathcal{J}$ qui est annulé par un polynôme $P(X)$ quadratique irréductible.

Dans ce cas, le polynôme $P(X)$ engendre une extension quadratique de la forme $k(\sqrt\Delta)$ et l'on dit alors que la courbe est à multiplication réelle par $\sqrt\Delta$.
\end{definition}

\subsection{Une famille de courbe à multiplication par \texorpdfstring{$\sqrt{2}$}{sqrt2}}

On reprend un résultat de Bending qui présente une famille universelle de courbe de genre 2 dont la Jacobienne dispose d'une multiplication réelle par $\sqrt2$ définie sur le corps de base et fixée par l'involution de Rosati.

Le théorème suivant est une conséquence immédiate du Théorème 4.1 de \citep{bending} :
\begin{theoreme}[\citet{bending}]
\label{bending}
Soit $C$ courbe de genre 2 définie sur $k$ par l'équation suivante :
$$C : y^2 = \Delta G_1(x)G_2(x)G_3(x)$$
avec
\begin{align*}
G_1(X) &= X^2 + 4s \\
G_2(X) &= X^2 - tX - \frac{1}{s} \\
G_3(X) &= X^2 - \frac{4s^2+1}{s^2t}X - \frac{1}{s}
\end{align*}
où $\Delta$, $s$ et $t$ sont des éléments non nuls de $k$.

Alors $C$ dispose d'une multiplication réelle par $\sqrt2$, notée $\epsilon$, définie sur $k$ et fixée par l'involution de Rosati. De plus, on a la relation suivante :
$$\epsilon = \pm \iota^{-1}\rho$$
où $\rho$ est l'isogénie de Richelot et $\iota$ est l'isomorphisme entre la courbe $C$ et sa duale de Richelot $\hat{C}$ définie par :
\begin{equation*}
\begin{array}{lrcl}
\iota :&C & \longrightarrow & \hat{C} \\
& (x,y) & \longmapsto & (\frac{2}{x},\frac{4y}{x^3})
\end{array}
\end{equation*}
%
%La duale de Richelot $\hat{C}$ a pour équation :
%$$\hat{C} : 4sy^2 = \Delta H_1(x)H_2(x)H_3(x)$$
%où
%\begin{align*}
%H_1(X) &= X^2 + \frac{1}{s} \\
%H_2(X) &= X^2 + 2stX - 4s \\
%H_3(X) &= X^2 + 2\frac{4s^2+1}{st}X - 4s
%\end{align*}
\end{theoreme}

On reprend maintenant ce que l'on a fait précédemment en spécialisant les résultats à cette famille de courbe. Un calcul de résultant donne facilement :
\begin{align*}
e_1 = -Res(G_1,G_2) &= -\frac{4s^3t^2 + 16s^4 + 8s^2 + 1}{s^2} \\
e_2 = -Res(G_2,G_3) &= \frac{(s^2t^2 - 4s^2 - 1)^2}{s^5t^2} \\
e_3 = -Res(G_1,G_3) &= -\frac{(st^2 + 4)(4s^2 + 1)^2}{s^3t^2}
\end{align*}

On cherche une condition pour que les éléments $e_1e_2$, $e_1e_3$ et $e_2e_3$ soient des carrés dans $k$; ceci pour pouvoir calculer explicitement l'isogénie de Richelot et un doublement rapide.

\begin{proposition}
Les éléments $e_1e_2$, $e_1e_3$ et $e_2e_3$ sont des carrés dans $k$ si et seulement si
$-\frac{4s^3t^2 + 16s^4 + 8s^2 + 1}{s}$, $-(st^2+4)$ sont des carrés dans $k$.
\end{proposition}

Une fois que l'on a calculé des racines carrées de $e_je_{j+1}$, on calcule les coefficients de la surface de Kummer $\mathcal{K}$ en utilisant les relations que l'on a trouvé précédemment mais spécialisé à cette famille de courbe; ce qui donne les égalités suivantes :
\begin{align*}
c_1+c_2+c_3+c_4 &= (64s^6t^2 + 16s^4t^4 + 128s^5t^2 - 16s^4t^2 + 64s^3t^2 \\
    & + 256s^4 - 4s^2t^2 + 8st^2 + 128s^2 + t^2 + 16)/(s^4t^2) \\
c_1+c_2-c_3-c_4 &= \frac{-(2s + 1)^2}{s^2}\epsilon_2\epsilon_3 \\
c_1-c_2+c_3-c_4 &= \frac{2(4s^2 - 1)}{s}\epsilon_1\epsilon_3 \\
c_1-c_2-c_3+c_4 &= \frac{2(4s^2 - 1)}{s}\epsilon_1\epsilon_2
\end{align*}

\begin{remarque}
Par rapport aux relations précédentes, on a divisé par le facteur projectif $\lambda$ qui n'est pas nécessaire pour faire les calculs.
\end{remarque}

%D'autre part, un calcul de discriminant montre que l'on a :
%\begin{align*}
%-disc(G_1) &= 16s \\
%-disc(G_2) &= -\frac{st^2 + 4}{s} \\
%-disc(G_3) &= -\frac{4s^3t^2 + 16s^4 + 8s^2 + 1}{t^2s^4}
%\end{align*}

%On en déduit alors le lemme suivant :
%\begin{lemme}
%Soit $G_1,G_2,G_3$ comme dans le théorème \ref{bending}, alors
%$e_1,e_2,e_3$ sont des carrés si et seulement si $-disc(G_1),-disc(G_2),-disc(G_3)$ sont des carrés.
%\end{lemme}
%
%On remarque alors un comportement différent selon que $-1$ est un résidu quadratique ou non. En effet, pour un polynôme $P(X)$ séparable quadratique, on sait que $P$ se factorise en facteurs linéaires si et seulement si $disc(P)$ est un carré; de plus, dans le cas contraire, $P$ est irréductible.
%
%Ce qui nous permet d'obtenir immédiatement le corollaire suivant :
%\begin{corollaire}
%Soit $G_1,G_2,G_3$ comme dans le théorème \ref{bending}.
%
%Si $-1$ est un résidu quadratique, alors $e_1,e_2,e_3$ sont des carrés si et seulement si $G_1,G_2,G_3$ se décomposent en facteurs linéaires.
%
%Si $-1$ n'est pas un résidu quadratique, alors $e_1,e_2,e_3$ sont des carrés si et seulement si $G_1,G_2,G_3$ sont irréductibles.
%\end{corollaire}
%
\begin{remarque}
On vérifie expérimentalement (pour la famille de Bending \ref{bending}) que le fait que $e_1e_2$, $e_1e_3$ et $e_2e_3$ soient des carrés est suffisant pour que les constantes $y_0$, $z_0$, $t_0$ soient définies sur le corps de base.
\end{remarque}

Le calcul de la $\sqrt{2}$ tel que l'exprime le théorème \ref{bending} est (semble t'il) plus complexe qu'un doublement par la présence de l'isomorphisme $\iota$ et on perd en efficacité en passant par celui-ci.
En effet, le problème est que malgré le fait que la courbe $C$ et sa duale de Richelot $\hat{C}$ soient isomorphes, la surface de Kummer duale $\hat{\mathcal{K}}$ peut avoir une équation différente de la surface de Kummer $\mathcal{K}$ ce qui nous oblige à passer par un isomorphisme non trivial entre ces deux surfaces.
Le calcul de cet isomorphisme semble plutôt compliqué et il semble que l'on n'aura pas de $\sqrt{2}$ rapide dans ce cas.

Expérimentalement, on trouve que la proportion des courbes dans la famille de Bending pour lesquelles on peut calculer explicitement une $\sqrt{2}$ rapide est d'environ 3\%; ce qui donne environ $0,03p^2$ courbes disposant d'une $\sqrt{2}$ rapide dans $\mathbb{F}_p$.

\begin{remarque}
La famille de Bending est intéressante puisqu'elle nous donne une proportion suffisante de courbe avec $\sqrt{2}$ rapide; de plus, le théorème de \citep{bending} peut éventuellement se généraliser en caractéristique $p$ pour montrer que cette famille reste une famille universelle de courbe avec $\sqrt{2}$ dans $\mathbb{F}_p$.

Cependant, il faut faire une trentaine d'essais avant de trouver une bonne courbe dans la famille, ce qui peut poser des problèmes d'efficacité lorsque l'on essaye de choisir une courbe sécurisée disposant d'une arithmétique rapide.

Il faudrait donc essayer de trouver une famille de courbe pour lesquelles la surface de Kummer duale ait la même équation que la surface de Kummer originale; en terme de thetas constantes, cela correspond au cas où $a^2 = b^2 + c^2 + d^2$.
\end{remarque}


\subsection{Utilisation de \texorpdfstring{$\sqrt{2}$}{sqrt2}}

D'après le théorème \ref{mulscal}, le calcul d'une multiplication scalaire par $n$ nécessite $c \log_2 n$ opérations dans le corps de base; où $c$ est une constante qui vaut approximativement $25$. On peut jouer sur le fait qu'une élévation au carré est légèrement moins couteuse qu'un produit ou qu'un produit par une petite constante est beaucoup moins couteux pour avoir des résultats plus précis dans certains cas.

L'objectif est maintenant d'améliorer la constante $c$ (indépendamment des considérations précédentes) en utilisant un endomorphisme rapide; dans le cas présent, utiliser la $\sqrt{2}$ que l'on s'est donné tant de mal à calculer.

Lorsque l'on restreint la $\sqrt{2}$ à un sous-groupe cyclique $G \cong \mathbb{Z}/N\mathbb{Z}$ de la Jacobienne $\mathcal{J}$, celle-ci agit sur $G$ comme une multiplication par un entier $\lambda$ tel que $\lambda^2 = 2 \mod N$.

Pour calculer la multiplication scalaire $nP$, on peut alors écrire $n$ sous la forme suivante :
$$ n = a + \lambda b \mod N$$

On voit donc immédiatement que pour un point $P \in G$, on peut calculer la multiplication scalaire par $n$ en utilisant l'égalité
$$nP = aP + b(\sqrt{2}P)$$

On peut penser que cette écriture nous demandera deux fois plus de travail puisqu'il y a deux multiplications scalaires suivie d'une addition; cependant, il existe des algorithmes de multiexponentiations rapides qui permettent de faire ce calcul en utilisant une boucle de longueur $max(\log_2 a, \log_2 b)$. D'une certaine manière, on effectue deux calculs pour le prix d'un.

Maintenant, il faut choisir $a$ et $b$ convenablement pour que cette approche nous offre effectivement un gain. Il se trouve que l'on peut choisir $a$ et $b$ de l'ordre de $\sqrt{n}$; ce qui permet de faire la multiexponentiation en une boucle de longueur $\frac{1}{2}\log_2 n$.

\citet{bernstein} présente une chaine d'addition en dimension 2 qui permet de calculer la multiexponentiation rapide sur la surface de Kummer dans une boucle qui nécessite un unique DBLDBLADD par itération et un calcul préliminaire de $(1+\sqrt{2})P$; où l'on note DBLDBLADD pour le résultat du calcul de $2P$, $P+Q$, $R+S$ à partir des points $P$, $Q$, $R$, $S$, $P-Q$ et $R-S$.

Un décompte précis du nombre d'opérations nécessaire dans le calcul de DBLDBLADD en utilisant des optimisations à la Gaudry donne la proposition suivante :
\begin{proposition}
Avec les formules de \citep{gaudry}, le calcul d'un DBLDBLADD nécessite $23$ produits et $13$ élévations au carré dans le corps de base.
\end{proposition}

Ce qui nous permet d'en déduire grâce à la discussion précédente le coût d'une multiplication scalaire en utilisant une chaine d'addition en dimension 2 et la $\sqrt{2}$ :
\begin{corollaire}
Avec les formules de \citep{gaudry}, le calcul d'une multiplication scalaire d'un point $P \in \mathcal{K}$ par un entier $n > 1$ nécessite $11,5\log_2 n$ produits, $6,5\log_2 n$ élévations au carré dans le corps de base et le calcul de $(1+\sqrt{2})P$.
\end{corollaire}

L'utilisation d'une chaine d'addition de dimension 2 à la Bernstein, lorsque l'on dispose d'une multiplication par $\sqrt{2}$, nous permet d'améliorer de 28\% la constante dans le théorème \ref{mulscal} au prix du calcul de $(1+\sqrt{2})P$.

On est encore une fois face à un compromis à faire, on optimise la boucle principale au prix d'un précalcul supplémentaire; le gain effectif sera donc énormément dépendant de la taille des paramètres et du cout réel du calcul de $(1+\sqrt{2})P$.

\section{Conclusion}
Ce stage %au sein de l'équipe \href{http://www.lix.polytechnique.fr/cryptologie/}{GRACE}
a été l'occasion de se familiariser aux sujets vaste et fascinant que sont la cryptographie et la théorie algorithmique des nombres. Nous avons essayé d'étendre (dans la mesure du possible) la classe de courbes qui admettent une arithmétique à la Gaudry et de donner des conditions suffisantes pour avoir une arithmétique rapide de genre 2.

Si l'on ne devait retenir qu'une seule chose de ces quelques pages, ce serait le slogan suivant :
\begin{center}
\huge{Une 2-isogénie diagonale induit une arithmétique rapide.}
\end{center}

Il reste encore quelques questions en suspens qui pourront être intéressantes pour le début d'une thèse; notamment, le calcul explicite de $1+\sqrt{2}$ s'est avéré plus subtil et compliqué que l'on aurait pu le penser à priori.

D'autre part, la piste d'un algorithme à la Schoof sur la surface de Kummer sera aussi un exercice intéressant qui occupera une partie de la thèse. Une étape importante sera de définir un analogue des polynômes de $l$-division sur la surface de Kummer qui permettent une représentation efficace de la $l$-torsion. Pour finir, on s'intéressera au cas des courbes à multiplication par $\sqrt{2}$ et l'on aura besoin de manière cruciale du calcul explicite de $1+\sqrt{2}$ que l'on aura effectué précédemment.

Nous nous sommes restreint au cas de la caractéristique impaire, en ce qui concerne la caractéristique 2 le modèle classique ne convient plus et il est nécessaire d'introduire un terme supplémentaire pour désingulariser la courbe. \citet{lubicz} ont étudié les modifications nécessaire pour avoir une arithmétique rapide à base de fonctions thêtas. Il serait intéressant de reprendre notre étude dans le cas de la caractéristique 2; ce cas étant particulièrement utile pour des applications cryptographiques dans les systèmes embarqués. On termine par l'énoncé de l'analogue du théorème \ref{mulscal} en caractéristique 2 :

%\begin{theoreme}[\citep{lubicz}]
%Multiplying by a scalar a point on a Kummer surface with non-zero coordinates costs
%9 squarings, 12 general multiplications, 3 multiplications by coordinates of the base point and 3 multiplications by constants that depend only on the Kummer surface, per bit of the scalar.
%\end{theoreme}

\begin{theoreme}[\citep{lubicz}]
La multiplication par un scalaire $n > 1$ d'un point $P$ à coordonnées non nulles sur la surface de Kummer nécessite $9\log_2 n$ élévations au carrés, $12\log_2 n$ produits, $3\log_2 n$ produits par les coordonnées du point $P$ et $3\log_2 n$ produits par des constantes qui dépendent uniquement de la surface de Kummer.
\end{theoreme}

L'énoncé est presque identique au théorème \ref{mulscal} en terme de constantes, l'avantage est cependant que le matériel est optimisé pour la caractéristique 2 et on peut espérer avoir un gain supplémentaire en utilisant cette spécificité.
%Pour finir, le problème du logarithme discret de genre 2 sera un fil directeur tout au long de la thèse.

\bibliographystyle{plainnat-fr}
\bibliography{ref}

\appendix

\section{Correspondance de Richelot}

%
%\section{Algorithme de Schoof}
%\emph{Détails d'implémentation} : On effectue tous les calculs modulo l'idéal de $l$-division par l'intermédiaire des bases de Gröbner; les performances de l'algorithme sont donc dépendantes de l'ordre monomial choisi. Dans le cas présent, l'ordre monomial le mieux adapté est l'ordre lexicographique avec $Y < X$; ce qui consiste lorsque l'on effectue une réduction modulo l'idéal, à d'abord remplacer les polynômes en $X,Y$ par des polynômes de degré $\leq 1$ en $Y$ en utilisant la relation $Y^2 = X^3 + aX + b$ puis à utiliser la relation $\Psi_l=0$ pour finalement obtenir un polynôme de degré $\leq 1$ en Y et de degré $< deg(\Psi_l)$ en X.
%
%\emph{Passage à la droite projective} : L'objectif consistant à adapter l'algorithme de Schoof pour des surfaces de Kummer en genre 2, on commence par effectuer une variante de l'algorithme de Schoof classique de genre 1 pour les variétés de Kummer de courbe elliptique qui n'est autre que la droite projective. La majeure partie de l'algorithme s'adapte facilement étant donné que l'on sait faire des multiplications scalaires sur la variété de Kummer en utilisant des chaines d'addition différentielle. Il n'y a qu'une opération que l'on ne sait pas faire sur la variété de Kummer; c'est une addition "pure" i.e. on ne sait pas calculer le point $(X^{p^2},Y^{p^2}) + p(X,Y)$ si l'on ne connait pas le point $(X^{p^2},Y^{p^2}) - p(X,Y)$. Bien entendu, on ne connait ni l'un ni l'autre à priori; cependant, a posteriori on veut tester l'égalité \eqref{polCarFrob}. On peut donc calculer le point $(X^{p^2},Y^{p^2}) - p(X,Y)$ en supposant l'égalité \eqref{polCarFrob}. Comme on ne sait pas à priori que l'égalité \eqref{polCarFrob} est vérifiée, on ne sait pas si le calcul que l'on vient d'effectuer est juste; cependant, pour le vérifier, il nous suffit de tester si les deux points obtenus vérifient les mêmes relations que les points $(X^{p^2},Y^{p^2}) + p(X,Y)$ et $(X^{p^2},Y^{p^2}) - p(X,Y)$. À savoir, on veut vérifier que l'on a bien
%\begin{align*}
%((X^{p^2},Y^{p^2}) + p(X,Y)) - ((X^{p^2},Y^{p^2}) - p(X,Y)) &= 2p(X,Y) \\
%((X^{p^2},Y^{p^2}) + p(X,Y)) + ((X^{p^2},Y^{p^2}) - p(X,Y)) &= 2(X^{p^2},Y^{p^2})
%\end{align*}
%Encore une fois, on n'a pas de vraie addition sur la variété de Kummer; on ne peut donc pas vérifier ces deux égalités directement. On peut cependant supposer la première égalité et vérifier que l'on obtient bien la seconde égalité. Si l'on a bien ces égalités alors on sait que l'égalité \eqref{polCarFrob} est vérifiée dans la surface de Kummer. Bien entendu, la variété de Kummer ne fait pas la différence entre $t'(X^p,Y^p)$ et $-t'(X^p,Y^p)$; au final, on obtient donc $\pm t \mod l$. On a effectué le compromis suivant : on a supprimé la coordonnée $y$ en passant à la surface de Kummer ce qui nous économise un certain nombre de calculs et évite l'usage des bases de Gröbner; la contrepartie étant que l'on ne connait pas $t \mod l$ entièrement, on ne connait que $\pm t \mod l$. Ce dernier point peut s'expliquer a posteriori d'une seconde manière; en effet, la variété de Kummer ne distingue pas la courbe elliptique de son twist quadratique et on sait que la trace du twist quadratique est l'opposé de la trace de la courbe d'origine. \emph{Le problème qui se pose est de savoir si l'on peut reconstituer la trace à partir de ces seules informations locales .} En effet, on a deux possibilités pour chaque nombre premier $l$ et on ne sait pas comment les recombiner sans informations supplémentaires, car toutes les combinaisons sont possibles d'après le lemme chinois.
%
%\section{Calcul de \texorpdfstring{$(1 \pm \phi)(P)$}{(1 +- sqrt2)P}}
%
%La première étape du calcul de la pseudo addition entre $P$ et $\phi(P)$ consiste à appliquer l'isogénie à $P$ et $\phi(P)$, ce qui donne les points $\phi(P)$ et $2P$ respectivement puis à multiplier les deux points coordonnées à coordonnées; la seconde étape du calcul consiste alors à appliquer une application linéaire, plus précisément, l'application linéaire donnée par la matrice suivante :
%\[ \left(
%  \begin{array}{ c c c c}
%     1 & 1 & 1 & 1\\
%     1 & 1 & -1 & -1\\
%     1 & -1 & 1 & -1\\
%     1 & -1 & -1 & 1\\
%  \end{array} \right)
%\]
%Arrivé à ce point, on obtient les coordonnées d'un point obtenu en multipliant $(1 \pm \phi)(P)$ coordonnées à coordonnées.
%
%On aimerait décomposer ce produit en deux morceaux pour obtenir les coordonnées des deux points $(1 + \phi)(P)$ et $(1 - \phi)(P)$ avec éventuellement une ambiguïté sur les deux points. On utilise un logiciel de calcul formel pour voir si on arrive à faire cette décomposition lorsque les coordonnées de P sont des indéterminées $P=(X,Y,Z,T)$, malheureusement les polynômes que l'on obtient sont irréductibles. On est peut-être trop optimiste de penser pouvoir factoriser ce polynôme sans prendre en compte l'équation de la surface de Kummer; on essaie donc de factoriser le polynôme en partant d'un point générique, autrement dit, à factoriser l'idéal engendré par l'équation de la surface de Kummer et ce polynôme. Cette fois encore le résultat n'est pas à la hauteur des espérances.
%On sait que pour un point d'ordre deux la pseudo-addition est bien définie, car alors les deux points sont égaux; cependant, on se retrouve à résoudre le système
%\begin{align*}
%X^2 &= \alpha  &Y^2 &= \beta \\
%Z^2 &= \gamma  &T^2 &= \delta
%\end{align*}
%Il n'y a aucun choix des racines carrées à privilégier, on n'a pas à déterminer entièrement la somme alors que la théorie nous dit qu'on devrait pouvoir en être capable. Le cas le plus flagrant de ce phénomène intervient lorsque que l'on calcule $P+0$, où $P=(x,y,z,t)$ et $0=(a,b,c,d)$ est le neutre; on obtient alors les équations :
%\begin{align*}
%X^2 &= x^2  &Y^2 &= y^2 \\
%Z^2 &= z^2  &T^2 &= t^2
%\end{align*}
%Le résultat est bien entendu censé être $(X,Y,Z,T)=(x,y,z,t)=P$; cependant, la formule permettant de calculer la pseudo-addition ne le permet pas. En effet, pour calculer $P+Q$ avec l'algorithme de pseudo-addition, on a besoin de connaître le point $P-Q$ (en dehors de $P$ et $Q$).
%\emph{Il pourrait être intéressant de trouver une formule qui permette de trouver le point $P \pm Q$ sans connaitre aucun des deux points $P+Q$ et $P+Q$.}
%Une telle formule existe pour la variété de Kummer d'une courbe elliptique qui correspond à une représentation de la courbe elliptique par la coordonnée $x$ uniquement (voir \citet{montgomery}).
%
%La première idée qui vient à l'esprit est de remonter les points $P$ et $Q$ dans la Jacobienne en $\pm P$ et $\pm Q$, d'utiliser l'addition sur la Jacobienne (qui est bien définie) et finalement de redescendre dans la surface de Kummer. Cette stratégie est un peu lourde à mettre en place pour les coordonnées thêtas mais en théorie on devrait pouvoir mener le calcul à bien. Le problème étant que l'on a les mêmes relations pour les fonctions thêtas dans la Jacobienne et dans la surface de Kummer; ce qui paraît étrange puisque l'on est censé avoir une addition bien définie sur la Jacobienne, mais les relations thêtas ne permettent pas de faire le calcul aussi facilement. \citet{cosset} donne une formule qui pourrait être utile à nos besoins; elle permet de calculer $\theta_i(2(z+z'))\theta_j(2(z-z'))$ en fonction des autres fonctions thêtas aux points $2z$ et $2z'$. On pourrait penser que les formules de \citet{gaudry} donnent la même chose; cependant, la différence fondamentale est que $i$ et $j$ peuvent être différents contrairement aux formules de Gaudry. En laissant fixe l'indice $j$, on peut utiliser le fait que la surface de Kummer est projective pour en déduire $\theta_i(2(z+z'))$. Le désavantage de cette formule est qu'il nous faut calculer toutes les fonctions thêtas; on dispose uniquement des fonctions thêtas au carré, ce qui paraît normal vu que la connaissance de toutes les fonctions thêtas nous donne un point sur la Jacobienne et il n'y aurait plus d'ambiguïté entre $\pm P$. Cette formule est donc une façon déguisée d'obtenir $P \pm Q$ en remontant d'abord dans la Jacobienne. Pour finir, l'idée la plus intéressante consiste à obtenir des formules similaires aux formules de Gaudry i.e. en passant par les fonctions thêtas duales, pour calculer $\theta_i(2(z+z'))\theta_j(2(z-z'))$; de telles formules existent. Par exemple, en prenant $j=1$, on obtient les formules suivantes :
%\[
%\begin{aligned}
%\vartheta_1(z+z')\vartheta_1(z-z') ={}& \Theta_1(2z)\Theta_1(2z') + \Theta_2(2z)\Theta_2(2z')\\
% &+ \Theta_3(2z)\Theta_3(2z') + \Theta_4(2z)\Theta_4(2z') \\
%\vartheta_2(z+z')\vartheta_1(z-z') ={}& \Theta_5(2z)\Theta_5(2z') + \Theta_6(2z)\Theta_6(2z') \\
% &+ \Theta_7(2z)\Theta_7(2z') + \Theta_8(2z)\Theta_8(2z') \\
%\vartheta_3(z+z')\vartheta_1(z-z') ={}& \Theta_9(2z)\Theta_9(2z') + \Theta_{10}(2z)\Theta_{10}(2z') \\
% &+ \Theta_{11}(2z)\Theta_{11}(2z') + \Theta_{12}(2z)\Theta_{12}(2z') \\
%\vartheta_4(z+z')\vartheta_1(z-z') ={}& \Theta_{13}(2z)\Theta_{13}(2z') + \Theta_{14}(2z)\Theta_{14}(2z') \\
% &+ \Theta_{15}(2z)\Theta_{15}(2z') + \Theta_{16}(2z)\Theta_{16}(2z') \\
%\end{aligned}
%\]
%L'avantage de ces formules est que l'on peut calculer la plupart des fonctions thêtas duales à partir des carrés des fonctions thêtas de départ que l'on connait en utilisant des formules analogues à celles présentes dans \citep{gaudry}. On ne peut pas calculer directement toutes les fonctions thêtas duales, car cela reviendrait à remonter sur la Jacobienne duale; en effet, on ne peut pas calculer les fonctions thêtas impaires par ces formules, car les thêtas constantes sont nulles.
%
%\section{Idéal de \texorpdfstring{$l$}{l}-division}
%L'idéal de $l$-division de la surface de Kummer correspond à un point de $l$-torsion générique i.e. la variété correspondante est la variété de $l$-torsion de la surface de Kummer. Il est engendré par l'équation de la surface de Kummer et les équations que l'on dérive de l'égalité $[l]P=0$.
%
%La multiplication par $2$ étant donnée relativement explicitement, on obtient facilement l'idéal de $2$-division; de même, les idéaux de $2^n$-division sont tout aussi facilement obtenus par des applications successives de la multiplication par $2$.
%
%Pour $l$ premier impair, la situation est plus délicate car on doit utiliser des pseudo-additions. En effet, la dernière étape de la pseudo-addition consiste en des divisions des coordonnées de $P-Q$; on travaille avec des coordonnées polynomiales et bien souvent on ne peut pas faire cette division directement. La première solution envisageable consiste à se rappeler que l'on travaille dans l'espace projectif et à remplacer les divisions par des multiplications. Cette solution a l'avantage d'être simple, la contrepartie étant qu'elle augmente le degré des polynômes. \emph{Malheureusement, il semblerait que la variété projective correspondante en utilisant cette solution soit de dimension $1$ projective donc cette solution n'est pas utilisable.}
%La seconde solution est d'utiliser les bases de Gröbner pour faire ladite division. Cette fois on n'augmente pas inutilement le degré des polynômes, mais on fait des calculs de base de Gröbner qui peuvent être couteux. \emph{Je n'ai pas encore testé cette solution, il me faut trouver comment on calcule les quotients; cependant, j'espère qu'elle donnera quelque chose d'intéressant pas comme la solution précédente.}
%
%Le degré $d(n)$ des polynômes de $n$-division vérifie la formule de récurrence suivante :
%$$d(n+1) = 2d(n) + 2 - d(n-1)$$
%On peut en déduire une relation simple pour les différences $d(n+1)-d(n)$ en fonction de $d(n)-d(n-1)$ :
%$$d(n+1)-d(n) = d(n) - d(n-1) + 2$$
%On voit donc que la suite $(d(n+1)-d(n))$ est une suite arithmétique de raison $2$ et de premier terme $d(2)-d(1) = 4 - 1 = 3$; on en déduit que $d(n+1)-d(n) = 3 + 2(n-1) = 2n + 1$.
%On a alors
%\begin{align*}
%d(n) &= (d(n) - d(n-1)) + (d(n-1) - d(n-2)) + ... + (d(2)-d(1)) + d(1) \\
% &= (2n-1) + (2n - 3) + ... + 1 \\
% &= n^2
%\end{align*}
%
%%\begin{equation}
%%    d(n) =
%%    \begin{cases}
%%       1 & \text{si } n = 1 \\
%%      4*d(n/2) & \text{si } n \text{ est pair} \\
%%      6*d((n-1)/2) + 1 & \text{si } n \text{ est impair} \\
%%    \end{cases}
%%\end{equation}
%
%Le temps de calcul des polynômes de $2^k$-division (qui devrait être les moins couteux à obtenir) explose rapidement; les polynômes de $8$-division s'obtiennent en $0.2$s alors que le calcul des polynômes de $16$-division prend $15$ minutes (pour $p=31$). On ne pourra pas raisonnablement aller plus loin.

\end{document}
