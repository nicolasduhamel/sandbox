\documentclass[a4paper]{article}

\usepackage[utf8]{inputenc}
\usepackage[T1]{fontenc}
\usepackage{lmodern}
\usepackage[francais]{babel}
\usepackage{amssymb}
\usepackage{amsmath}
\usepackage{amsthm}
\usepackage{hyperref}
\usepackage{xcolor}\hypersetup{linkbordercolor=green}
%\usepackage{graphics}
%\usepackage{enumerate}
\usepackage{epigraph}

% Theorem Styles
\newtheorem{theoreme}{Theorème}[section]
\newtheorem{lemme}[theoreme]{Lemme}
\newtheorem{proposition}[theoreme]{Proposition}
\newtheorem{corollaire}[theoreme]{Corollaire}
% Definition Styles
\theoremstyle{definition}
\newtheorem{definition}{Definition}[section]
\newtheorem{exemple}{Exemple}[section]
\theoremstyle{remark}
\newtheorem{remarque}{Remarque}
% Equation numerotation
\numberwithin{equation}{section}

\begin{document}

\title{Présentation magistère}
\author{N. Du Hamel}
\maketitle

\epigraph{Assume your adversary is capable of one trillion guesses per second.}{\textit{Edward Snowden}}

De nos jours, les concepts d’authentification et de confidentialité d'un point de vue informatique sont de plus en plus utilisé dans notre société moderne. On peut citer par exemple les cartes bancaires ou les sites web sécurisé avec le célèbre protocole HTTPS. Notre vie privé est de plus en plus mis en danger par certaines entités qui possèdent des capacités de collectes et de traitements de données inimaginables; comme le montre les récentes révélations à propos de la NSA et de ces nombreux programmes de collectes de communication à travers le monde.

\section{Motivation cryptographique}

On commence par donner une rapide introduction à quelques concepts clés de cryptographie moderne et on donne les principales motivations qui seront un guide lors du stage.

\subsection{Diffie-Hellman}
En 1976, Whitfield Diffie et Martin Hellman \cite{diffie-hellman} ont introduit le concept de cryptographie à clé publique aussi appelé cryptographie asymétrique. Contrairement à la cryptographie symétrique, où les deux parties cherche à communiquer en ayant au préalable convenu d'une clé secrète, chaque entité dispose d'une clé publique et d'une clé privée. La clé publique est mise à disposition sans restriction, l'expéditeur l'utilise pour chiffrer son message tandis que le destinataire utilise sa clé privée pour déchiffrer le message. D'autre part, l'expéditeur a la possibilité de signer le message avec sa clé privée; le destinataire utilise alors la clé publique de l'expéditeur pour vérifier l'authenticité du message.

Les implémentations de protocole à clé publique sont plus lent que leurs homologues symétrique; c'est pourquoi en pratique, la cryptographie à clé publique est souvent utilisée dans le seul but de partager une clé secrète qui pourra ensuite servir de base pour des protocoles symétrique qui sont plus rapide.

On se contentera de décrire le protocole d'échange de clés Diffie-Hellman; le but étant de donner une motivation au problème logarithme discret et non pas de faire un état de l'art de la cryptographie à clé publique.

Les deux protagonistes Alice et Bob désirent partager une clé secrète par le biais d'un canal de communication sans que Eve qui écoute les communications ne puisse avoir connaissance de ladite clé.

Alice et Bob commence par se mettre d'accord sur un groupe fini monogène $G \cong \mathbb{Z}/N\mathbb{Z}$ et un générateur $g$, ces deux paramètres peuvent être diffusé sur le canal de manière publique sans compromettre la sécurité du protocole. Alice choisit secrètement un entier $a \mod N$ et envoie $g^a$ à Bob. Bob choisit secrètement un entier $b \mod N$ et envoie $g^b$ à Alice. Alice et Bob disposent alors de la clé secrète commune $g^{ab}$; en effet, Alice reçoit $g^b$ de la part de Bob et calcule $(g^b)^a = g^{ab}$; Bob reçoit $g^a$ de la part d'Alice et calcule $(g^a)^b = g^{ab}$. Ni Alice ni Bob ne connait la clé secrète de son homologue et pourtant ils ont réussi à se mettre d'accord sur une clé secrète en combinant les deux clés secrète. Eve qui écoutait le canal de communication dispose de $g^a$ et $g^b$; si elle veut connaitre la clé secrète commune d'Alice et Bob, Eve doit calculer $g^{ab}$ à partir de $g^a$ et $g^b$.

\begin{definition}
On appelle \emph{problème de Diffie-Hellman} le problème consistant à trouver $g^{ab}$ à partir de $g^a$ et de $g^b$.

Soit $h \in G$. On appelle \emph{logarithme discret de h dans la base g} un représentant de la classe $a \mod N$ tel que $h=g^a$, on note $\log_g h = a$.
\end{definition}

On remarque que si l'on sait résoudre le problème de logarithme discret alors on sait aussi résoudre le problème de Diffie-Hellman; on calcule le logarithme discret de $g^a$ et $g^b$ à savoir $a \mod N$ et $b \mod N$ puis on calcule $g^{ab}$. Le problème de Diffie-Hellman n'est donc pas plus difficile que le problème du logarithme discret.

\subsection{Logarithme discret}
Le protocole de Diffie-Hellman fait l'hypothèse que le calcul de logarithme discret dans certains groupes est difficile (du point de vue algorithmique). Cette hypothèse n'est pas vérifiée pour tous les groupes que l'on peut considérer; dans $(\mathbb{Z}/N\mathbb{Z},+)$, le calcul du logarithme discret est facile. Si $g$ est un générateur de $\mathbb{Z}/N\mathbb{Z}$, on sait qu'il est premier à $N$ et en utilisant l'algorithme d'Euclide étendu on peut calculer un inverse $g^{-1}$ de $g \mod N$. Le logarithme discret de $x$ est alors $g^{-1}x$, la complexité du calcul du logarithme discret dans $(\mathbb{Z}/N\mathbb{Z},+)$ est $O(\log^2N)$.

La difficulté du logarithme discret est indépendante du choix d'un générateur. En effet, soit $h,g \in G$ deux générateurs et soit $x \in G$, alors on a :
$$\log_h x = (\log_g x)(\log_g h)^{-1}$$
On en déduit que si l'on sait calculer efficacement le logarithme discret dans la base $g$ alors on sait aussi calculer efficacement le logarithme discret dans la base $h$.

Cependant, la difficulté du logarithme discret n'est pas invariant par isomorphisme. En effet, le calcul du logarithme discret est considérablement plus difficile dans le groupe multiplicatif d'un corps fini $\mathbb{F}^*_q$ que dans le groupe additif $\mathbb{Z}/(q-1)\mathbb{Z}$ qui lui est isomorphe. La raison de cette non invariance par isomorphisme est que ces isomorphismes ne sont pas effectif. Le calcul de l'isomorphisme entre $\mathbb{Z}/(q-1)\mathbb{Z}$ et $\mathbb{F}^*_q$ est en fait équivalent au problème du logarithme discret dans $\mathbb{F}_q^*$.

La méthode naïve pour résoudre le logarithme discret consiste à tester tous les exposants les uns après les autres; la complexité de cet algorithme est en $O(N)$, ce qui est exponentiel en la taille de $N$ i.e. exponentiel par rapport à $\log N$. Une variante de l'algorithme $\rho$ de Pollard \cite{pollard} permet de résoudre le logarithme discret en $O(\sqrt{N})$ opérations; si l'on cherche le logarithme discret de $h$ en base $g$ l'algorithme $\rho$ de Pollard cherche des collisions sous la forme $h^{\alpha_1}g^{\beta_1} = h^{\alpha_2}g^{\beta_2}$ pour en déduire le logarithme discret.

L'algorithme de Pohlig-Hellman \cite{pohlig-hellman} permet de réduire la difficulté du logarithme discret dans le cas où l'ordre $N$ du groupe $G$ est composé. L'algorithme consiste à résoudre plusieurs logarithmes discrets dans des groupes plus petits puis à recomposer le logarithme dans $G$.

Supposons pour simplifier et expliquer l'idée générale de la réduction que l'ordre $N$ de $G$ est un produit de deux nombres premiers $p$ et $q$ distincts. Si l'on note $G[p]$ (resp $G[q]$) la composante $p$-primaire (resp $q$-primaire) de $G$ alors le calcul du logarithme discret dans $G$ se ramène à $2$ calculs de logarithmes discrets dans les sous-groupes primaires par l'application suivante :
\begin{equation*}
\begin{array}{lcl}
G & \longrightarrow & G[p] \times G[q] \\
h & \longmapsto & (h^q, h^p)
\end{array}
\end{equation*}
On supposant que l'on connait les logarithmes discrets $x_1$ (resp $x_2$) de $h^q$ (resp $h^p$) dans la base $g^q$ (resp $g^p$) alors on peut calculer le logarithme discret de $h$ dans le groupe $G$ en utilisant une relation de Bézout entre $p$ et $q$. En effet, supposons que l'on ait le relation de Bézout suivante :
$$u_1 q + u_2 p = 1$$
On en déduit alors une expression de $h$ en fonction de $h^p$ et $h^q$ :
\begin{align*}
h &= (h^q)^{u_1}(h^p)^{u_2} \\
  &= (g^q)^{x_1u_1}(g^p)^{x_2u_2}
\end{align*}
Ce qui nous permet d'en déduire le logarithme discret $x$ de $h$ en base $g$ par la formule suivante :
$$x = x_1 u_1 q + x_2 u_2 p$$

Si on applique simultanément l'algorithme de Pohlig-Hellman pour réduire le logarithme discret dans $G$ en des logarithmes discrets dans des groupes d'ordres premiers et l'algorithme $\rho$ de Pollard pour résoudre le logarithme discret dans les groupes d'ordres premiers; alors on trouve un algorithme générique qui a une complexité $O(\sum{\alpha_i\sqrt{p_i}}+k\log^2 N)$

Si on considère que le groupe $G$ est une boîte noire (groupe générique) i.e. que les seules opérations que l'on peut faire sont des multiplications d'éléments de $G$, l'inversion d'un élément et un test d'égalité entre deux éléments, alors on ne peut pas faire mieux.

\begin{theoreme}[Shoup \cite{shoup}]
Soit $G$ un groupe générique d'ordre $N$, notons $p$ le plus grand facteur premier de $N$. La résolution du problème du logarithme discret dans $G$ nécessite au moins $\Omega(\sqrt{p})$ opérations. 
\end{theoreme}

\subsection{Groupes cryptographique}
Le paragraphe précédent suggère que la sécurité du protocole Diffie-Hellman repose (en partie) sur la difficulté du problème du logarithme discret. L'objectif est donc d'arriver à trouver des groupes dans lesquels le problème du logarithme discret est difficile.

D'un point de vue historique, les premiers groupes qui ont été suggéré pour des applications en cryptographie sont les groupes multiplications d'un corps fini $\mathbb{F}^*_q$. On connait des attaques sur le logarithme discret dans ces groupes qui sont plus efficace que les approches génériques. La plupart de ces attaques reposent sur des méthodes de cribles qui ont été originellement conçues pour la factorisation d'entier et elles ont ensuite été adaptées pour le calcul du logarithme discret dans des corps finis. La complexité de ces algorithmes est sous-exponentielle $L_q[1/3;c]$ où l'on note
$$L_n[\alpha;c] = \exp(c(\log n)^\alpha(\log \log n)^{1-\alpha})$$

L'utilisation de courbes elliptiques et hyperelliptiques a été suggéré par Neal Koblitz \cite{koblitz1} \cite{koblitz2} en 1985 et 1989 respectivement. A l'heure actuelle, on ne connait d'algorithme qui arrive à faire mieux que la borne de Shoup pour des courbes elliptiques générales ou pour des courbes hyperelliptiques de genre 2 générales; même si on connaît des attaques spécifiques sur des courbes particulières, comme les courbes super-singulières par exemple.

\begin{definition}
Soit $\mathbb{F}_q$ un corps fini.
On appelle \emph{courbe elliptique} ( resp. \emph{courbe hyperelliptique de genre 2} ) sur $\mathbb{F}_q$ une courbe algébrique projective dont un modèle affine est de la forme
$$C : y^2 = F(x)$$
où $F(X)$ est un polynôme de degré 3 (resp. de degré 5) à coefficients dans $\mathbb{F}_q$ et sans facteur carré.
\end{definition}

L'ensemble des points sur $\mathbb{F}_q$ d'une courbe elliptique, noté $E(\mathbb{F}_q)$, est naturellement muni d'une structure de groupe abélien. Ce qui permet d'utiliser des groupes provenant de la géométrie algébrique pour des applications en cryptographie.

L'ensemble des points d'une courbe hyperelliptique n'est quand à lui pas naturellement muni d'une structure de groupe; cependant, si l'on considère des paires de points alors on peut munir cet ensemble d'une structure de groupe. On appelle \emph{Jacobienne} le groupe ainsi construit.

\subsection{Comptage de points}
La difficulté du logarithme discret repose (en partie) sur les propriétés arithmétiques de l'ordre du groupe que l'on considère pour des applications cryptographiques. Dans le cas des courbes elliptiques, l'ordre du groupe sous-jacent correspond au nombre de point sur la courbe elliptique dans $\mathbb{F}_q$; d'où l'importance d'avoir un algorithme efficace pour compter le nombre de point sur une courbe elliptique.

Schoof \cite{schoof} découvre en 1985 le premier algorithme polynomial de comptage de point sur une courbe elliptique; l'algorithme utilise de manière cruciale les polynômes de divisions et peut être améliorer si l'on arrive à trouver une factorisation partielle de ces polynômes, ce qui est fait par Atkin \cite{atkin} et Elkies \cite{elkies}.

Soit $E$ une courbe elliptique donnée par l'équation de Weierstass $$E : y^2 = x^3 + ax + b$$
On cherche à calculer le nombre de points $\mathbb{F}_p$-rationnel de la courbe $E$, que l'on notera $\#E(\mathbb{F}_p)$; pour ce faire, on va calculer $\#E(\mathbb{F}_p)$ modulo $l$ pour plusieurs nombres premiers $l \neq p$ et utiliser le lemme Chinois pour retrouver $\#E(\mathbb{F}_p)$. D'après la borne de Hasse, on a
$$p+1 - 2\sqrt{p} \leq \#E(\mathbb{F}_p) \leq p+1 + 2\sqrt{p}$$
La largeur de l'intervalle dans la borne de Hasse étant $4\sqrt{p}$, il suffit de faire les calculs de $\#E(\mathbb{F}_p) \mod l$ pour suffisamment de $l$ pour que l'on ait la relation $\prod_l l > 4\sqrt{p}$ et ainsi retrouver $\#E(\mathbb{F}_p)$ par le lemme Chinois.

Le morphisme de Frobenius $\pi : E \longrightarrow E$ vérifie la relation
$$\pi^2 - t\pi + pId = 0$$
où $t$ vérifie $\#E(\mathbb{F}_p) = p + 1 - t$. Pour trouver $t \mod l$, on teste les égalités
$$\pi^2(x,y) + p(x,y) = t'\pi(x,y), \quad t' = 0,1,...,l-1$$
pour tout point $(x,y)$ de $l$-torsion dans la clôture algébrique $\bar{\mathbb{F}}_p$. Le sous groupe de $l$-torsion
$$E[l] = \{P \in E(\bar{\mathbb{F}}_p), lP = 0\}$$
est défini sur une extension finie de $\mathbb{F}_p$; cependant, ce serait trop couteux de faire le calcul de tout les point de $l$-torsion. On va plutôt faire les calculs avec un point générique de $l$-torsion. Les polynômes de $l$-division $\Psi_l \in \mathbb{F}_p[X]$ sont des polynômes dont les racines sont exactement les coordonnées $x$ des points de $l$-torsion. On considère l'algèbre $$\mathbb{F}_p[X,Y]/(\Psi_l,Y^2-X^3-aX-b)$$
et l'on dira alors que le point $(X,Y)$ que l'on considère au dessus de cette algèbre est un point générique de $l$-torsion; en effet, la variété engendré par l'idéal $(\Psi_l,Y^2-X^3-aX-b)$ dans la clôture algébrique est exactement la $l$-torsion $E[l]$.

En résumé, l'algorithme de Schoof consiste à tester les égalités suivantes
\begin{equation}
(X^{p^2},Y^{p^2}) + (p \mod l)(X,Y) = t'(X^p,Y^p), \quad t' = 0,1,...,l-1
\end{equation}
modulo l'idéal $(\Psi_l,Y^2-X^3-aX-b)$ qui n'est valable que pour un seul $t'$ ce qui nous donne $t \mod l$; faire ce calcul pour suffisamment de $l$ et appliquer le lemme chinois pour trouver $t$.

\subsection{Courbes de genre 2}
Une \emph{courbe hyperelliptique de genre 2} est une courbe algébrique de la forme
$$C : y^2 = F(X) = f_5X^5 + ... + f_0$$
où $F(X)$ est un polynôme de degré 5 et sans facteur carré.

Une telle courbe est munie d'une involution $\iota : (x,y) \longmapsto (x,-y)$. On dira que $(x,-y)$ est le conjugué de $(x,y)$. Il y a 6 points particuliers sur la courbe qui sont invariants par l'involution, on les appelle \emph{points de Weierstrass}; ils sont de la forme $(\theta,0)$ où $\theta$ est l'une des 5 racines de $F(X)$ auquel on rajoute le point à l'infini.

\begin{remarque}
On peut montrer que toute courbe de genre 2 est birationnelle à une courbe de cette forme.
\end{remarque}

D'après les bornes de Hasse-Weil, on obtient l'ordre de grandeur du nombre de point sur la jacobienne d'une courbe de genre 2 $$\#J_C(\mathbb{F}_p) \approx p^2$$

L'idée d'utiliser des courbes de genre $2$ vient du fait que si l'on veut un groupe du même ordre de grandeur que pour une courbe elliptique, alors on peut se contenter d'utiliser un corps fini de taille plus petite et donc des clés de taille plus petite. Par exemple, pour avoir un groupe d'ordre approximativement $2^{256}$; il suffit dans le cas d'une courbe elliptique d'avoir $log_2(p) \approx 256$. Alors que pour avoir une jacobienne du même ordre de grandeur, il suffit d'avoir $log_2(p) \approx 128$.

L'ordre du groupe $N$ mesure la \emph{sécurité} que l'on peut espérer en utilisant ce groupe dans un protocole cryptographique. En effet, en supposant que le groupe est une boite noire pour lequel on n'a aucune information supplémentaire; alors le meilleur algorithme connu pour résoudre le logarithme discret est en $O(\log N + \sqrt{p})$, où $p$ est le plus grand facteur premier dans l'ordre du groupe. On voit donc que le temps nécessaire pour casser un protocole à base de courbes hyperelliptiques dépend du plus grand facteur premier de l'ordre du groupe, c'est pour cette raison que l'on s'intéresse le plus souvent à des groupes dont l'ordre est proche d'être un nombre premier (i.e. $N/p$ est petit).

\section{Surface de Kummer}

\subsection{Fonctions thêtas}
Les fonctions thêtas de Riemann sont une famille de fonctions holomorphes indexées par le demi-espace de Siegel $\mathfrak{H}_2$ des matrices de $M_2(\mathbb{C})$ tels que la partie imaginaire soit une matrice définie positive. Les fonctions thêtas sont alors définies comme des translatés à un facteur exponentiel prés des fonctions thêtas de Riemann.

\begin{definition}
Soit $\Omega$ une matrice de $\mathfrak{H}_2$, la fonction thêta de Riemann associé à $\Omega$ est la fonction holomorphe de $\mathbb{C}^2$ dans $\mathbb{C}$ définie par
$$\theta(z,\Omega) = \sum_{n \in \mathbb{Z}^2}{exp(i\pi n^T \Omega n + 2i\pi n^T z)}$$
pour $z \in \mathbb{C}^2$; on vérifiera que la série est bien convergente grâce à la condition sur $\Omega$, ce qui permet de conclure que la fonction $\theta$ est bien holomorphe.

Soit $a,b \in \mathbb{Q}^2$, la fonction thêta de caractéristique $(a,b)$ est définie par
$$\theta[a,b](z,\Omega) = exp(i\pi a^T\Omega a + 2i\pi a^T(z+b))\theta(z + \Omega a + b, \Omega)$$
\end{definition}

Dans la suite, on s'intéressera principalement aux fonctions thêtas avec caractéristiques dont les coordonnées sont dans $\{0,1/2\}$, ce qui nous donne 16 fonctions thêtas, on suit Gaudry \cite{gaudry} pour la numérotation. De plus, on appellera \emph{thêta constante} l'évaluation de ces fonctions thêtas au point $z=(0,0)$.

\subsection{Surface de Kummer}

\begin{definition}
La surface de Kummer $K$ associé aux fonctions thêtas est définie comme l'image de l'application suivante :
\begin{equation*}
\begin{array}{lrcl}
\phi :&\mathbb{C}^2 & \longrightarrow & \mathbb{P}^3(\mathbb{C}) \\
& z & \longmapsto & (\theta_1(z),\theta_2(z),\theta_3(z),\theta_4(z))
\end{array}
\end{equation*}
\end{definition}

La surface de Kummer $K$ est une variété projective définie par une équation de la forme suivante; voir Gaudry \cite{gaudry} :
$$X^4+Y^4+Z^4+T^4+2EXYZT-F(X^2T^2+Y^2Z^2)-G(X^2Z^2+Y^2T^2)-H(X^2Y^2+Z^2T^2)=0$$
où l'on note $(X:Y:Z:T)=(\theta_1(z):\theta_2(z):\theta_3(z):\theta_4(z))$ les coordonnées projectives sur $K$.

L'application $\phi$ est $\mathbb{Z}^2 + \Omega\mathbb{Z}^2$-périodique et définie donc une application depuis le tore complexe $\mathbb{C}^2/(\mathbb{Z}^2+\Omega\mathbb{Z}^2)$. Cependant, l'application $\phi$ n'est pas bijective; en effet, les 4 premières fonctions thêtas sont paires donc $\phi$ envoie deux points opposés sur le même point dans la surface de Kummer. On peut montrer que $K$ est isomorphe au tore complexe modulo ${\pm 1}$; autrement dit, on peut voir un point sur la surface de Kummer sous la forme $\pm P$ où $P$ est un point du tore complexe. Cette remarque nous permet de voir que l'on n'a pas tout à fait une structure de groupe sur la surface de Kummer mais on a une structure qui s'en approche le plus possible que l'on appellera pseudo-addition. En effet, si l'on dispose de deux points $\pm P$ et $\pm Q$ sur la surface de Kummer alors on ne peut pas faire la différence entre $P+Q$ et $P-Q$. Cependant, si l'on dispose de $P,Q$ et $P-Q$ alors on peut calculer $P+Q$. D'autre part, il est possible de calculer $2P$ uniquement avec la connaissance de $P$ et en utilisant la pseudo-addition on remarque que la multiplication scalaire $nP$ est bien définie.


\bibliographystyle{unsrt}
\bibliography{ref}
\end{document}
