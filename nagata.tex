\documentclass{article}

\usepackage[utf8]{inputenc}
\usepackage[T1]{fontenc}
\usepackage[francais]{babel}
\usepackage{amssymb}
\usepackage{amsmath}
\usepackage{amsthm}
\usepackage{graphics}
\usepackage{enumerate}

\newtheorem{proposition}{Proposition}
\newtheorem{definition}{Définition}
\newtheorem{theoreme}{Théorème}
\newtheorem{lemme}{Lemme}


\begin{document}

\title{Anneau noethérien de dimension infinie}
%\author{Duhamel Nicolas}
\maketitle

L'exemple suivant est dû à Nagata.
\begin{proposition}
Soit $k$ un corps et $A = k[T_0, T_1, ...]$ l'anneau des polynômes à une infinité de variables sur $k$. Soit $(m_n)_{n \geq 0}$ une suite strictement croissante d'entiers naturels tel que $m_{n+1} - m_n$ diverge, on note $P_n$ l'idéal premier engendré par les variables $(T_j)_{m_n \leq j < m_{n+1}}$ et $S$ la partie multiplicative $A-\cup_{n \geq 0}P_n$. Alors $S^{-1}A$ est un anneau noethérien de dimension infinie.
\end{proposition}

La première étape est de comprendre les idéaux maximaux de $S^{-1}A$.
On sait que les idéaux propres de $S^{-1}A$ sont de la forme $S^{-1}I$ où $I$ est un idéal de $A$ contenu dans $\cup_{n \geq 0}P_n$.

\begin{lemme}
Soit $I$ un idéal de $A$ contenu dans la réunion des $P_n$. Alors il existe un entier $n \geq 0$ tel que $I$ soit contenu dans $P_n$.
\end{lemme}

En effet, supposons que $I \neq (0)$ et soit $f \in I$ non nul. On note $P^{(1)}, ..., P^{(k)}$ les idéaux premiers de la forme $P_n$ qui sont engendrés par des variables qui apparaissent dans l'expression de $f$; ce sont exactement les idéaux $P_n$ tel que $f$ appartient à $P_n$. D'autre part, soit $g \in I$ non nul; on a $f+g \in I \subset \cup_{n \geq 0}P_n$. Si on suppose que $g \not\in \cup_{1 \leq i \leq k}P^{(i)}$ alors il existe un terme de $g$ qui n'est dans aucun des idéaux $P^i$; on a deux monômes de $f+g$ contenant des variables indépendantes qui sont contenu dans des idéaux $P_n$ distincts. On en déduit que $f+g \not\in \cup_{n \geq 0}P_n$; ce qui est une contradiction. Ce qui montre que $g \in \cup_{1 \leq i \leq k}P^{(i)}$; donc $I \subset \cup_{1 \leq i \leq k}P^{(i)}$. Le lemme d'évitement montre alors que $I$ est contenu dans l'un des idéaux $P^{(i)}$ qui est de la forme $P_n$.

D'après le lemme, on en déduit que les idéaux maximaux de $S^{-1}A$ sont exactement les idéaux de la forme $S^{-1}P_n$.
\begin{lemme}
Pour $n \geq 0$, l'anneau local $(S^{-1}A)_{P_n}$ est noethérien.
\end{lemme}

Les variables $T_j$ indépendantes de $T_{m_n}, ..., T_{m_{n+1}-1}$ n'appartiennent pas à $P_n$ qui correspond à l'idéal maximal de l'anneau local $(S^{-1}A)_{P_n}$; donc sont inversibles dans ce localisé. D'autre part, l'anneau $(S^{-1}A)_{P_n}$ est isomorphe à $A_{P_n}$ qui est un anneau de polynômes en $T_{m_n}, ..., T_{m_{n+1}-1}$ à coefficients dans le corps $k(T_j)$ où $j$ décrit les entiers positifs privé de $m_n, ..., m_{n+1}-1$. D'après le théorème de la base de Hilbert, l'anneau $(S^{-1}A)_{P_n}$ est noethérien.

On peut maintenant passer à la preuve du fait que l'anneau $S^{-1}A$ est noethérien. Soit $(S^{-1}I_j)_{j \geq 0}$ une suite d'idéaux propres de $S^{-1}A$. On suppose sans perte de généralité que $S^{-1}I_0 \neq (0)$, soit $f \in S^{-1}I_0$ non nul. On a déjà vu que $f$ n'appartient qu'à un nombre fini d'idéaux $S^{-1}P^{(1)}, ..., S^{-1}P^{(k)}$ de la forme $S^{-1}P_n$. Pour tout $j \geq 1$, l'idéal $S^{-1}I_j$ est contenu dans un idéal maximal de la forme $S^{-1}P_n$. On a $f \in S^{-1}I_0 \subset S^{-1}I_j \subset S^{-1}P_n$; on en déduit que $f$ appartient à $S^{-1}P_n$, ce dernier est donc l'un des idéaux $S^{-1}P^{(i)}$.
On a alors
\begin{equation*}
f \in S^{-1}I_0 \subset ... \subset S^{-1}I_j \subset ... \subset \cup_{1 \leq i \leq k}S^{-1}P^i
\end{equation*}
On localise maintenant cette chaine d'idéaux par rapport à la partie multiplicative $S^{-1}A-\cup_{1 \leq i \leq k}S^{-1}P^{(i)}$; comme précédemment les variables $T_j$ qui n'appartiennent pas à des idéaux $P^{(i)}$ sont inversibles car ne sont pas contenu dans $\cup_{1 \leq i \leq k}P^{(i)}$ et donc engendrent le localisé. On en déduit que ce localisé est un anneau de polynômes à un nombre fini de variables sur un corps; donc la chaine dans le localisé est de longueur finie. On en déduit que la suite d'idéaux $(S^{-1}I_j)$ est stationnaire; donc $S^{-1}A$ est noethérien.

Pour finir, il ne reste plus qu'à prouver que $S^{-1}A$ est de dimension infinie. Or la chaine
\begin{equation*}
S^{-1}(T_{m_n}) \subset S^{-1}(T_{m_n}, T_{m_n+1}) \subset ... \subset S^{-1}(T_{m_n}, ..., T_{m_{n+1}-1})
\end{equation*}
est de longueur $m_{n+1} - m_n$ qui tend vers l'infini. Comme la dimension est le supremum des longueurs des chaines d'idéaux premiers; on en déduit le résultat.
\end{document}