\documentclass{article}

\usepackage[utf8]{inputenc}
\usepackage[T1]{fontenc}
\usepackage[francais]{babel}
\usepackage{amssymb}
\usepackage{amsmath}
\usepackage{amsthm}
\usepackage{graphics}
\usepackage{enumerate}

% Theorem Styles
\newtheorem{theoreme}{Theorème}[section]
\newtheorem{lemme}[theoreme]{Lemme}
\newtheorem{proposition}[theoreme]{Proposition}
\newtheorem{corollaire}[theoreme]{Corollaire}
% Definition Styles
\theoremstyle{definition}
\newtheorem{definition}{Definition}[section]
\newtheorem{exemple}{Exemple}[section]
\theoremstyle{remarque}
\newtheorem{remarque}{Remarque}

%\newtheoremstyle{dotless}{}{}{itshape}{}{bfseries}{}{ }{}
%\theoremstyle{dotless}


\begin{document}

\title{Lemme fondamental d'Oka}
\date \today
\author{Duhamel Nicolas}
\maketitle

L'objectif est de démontrer un lemme d'annulation pour les faisceaux cohérents sur un domaine cylindrique convexe. On admettra le lemme de Cartan sur la décomposition des applications holomorphe à valeurs matricielles, que l'on utilisera pour démontrer le lemme de fusion. Cela nous permettra d'en déduire l'existence de syzygies d'Oka et finalement le lemme fondamental d'Oka.

\section{Lemme de fusion}
Soit $\mathcal{F}$ un faisceau cohérent sur un domaine $\Omega \subset \mathbb{C}^n$. A partir de systèmes générateurs finis de $\mathcal{F}$ sur $E'$ et $E''$, on veut construire un système générateur fini de $\mathcal{F}$ sur $E' \cup E''$.

On va supposer que $E'$ et $E''$ sont des produits de segments réels et on dira que ce sont des cubes. On supposera de plus que $E'$ et $E''$ sont des cubes adjacents, c'est à dire qu'ils peuvent s'écrire sous la forme $E'=F \times E'_n$ et $E''=F \times E''_n$ où $F$ est un cube dans $\mathbb{C}^{n-1}$, $E'_n$ et $E''_n$ sont des cubes de $\mathbb{C}$ et $e=E'_n \cap E''_n$ est un segment non vide.

\begin{lemme}[Décomposition matricielle de Cartan]
Il existe un voisinage $V_0 \subset GL_p(\mathbb{C})$ de l'identité $I_p$ tel que pour toute application holomorphe $\hat{A} : U \to V_0$ sur un voisinage $U$ de $F \times e$, il existe des applications holomorphes $A' : U' \to GL_p(\mathbb{C})$ et $A'' : U'' \to GL_p(\mathbb{C})$ où $U'$ et $U''$ sont des voisinages de $E'$ et $E''$ tel que $\hat{A} = (A')^{-1}A''$ sur $U' \cap U''$.
\end{lemme}

\begin{lemme}[Lemme de fusion]
On suppose qu'il existe un nombre fini de sections $\sigma'_j \in \mathcal{F}(U')$ pour $1\leq j \leq p'$ et $\sigma''_j \in \mathcal{F}(U'')$ pour $1\leq j \leq p''$ qui engendre $\mathcal{F}$ sur $U'$ et $U''$. De plus, on suppose qu'il existe des fonctions holomorphes $a_{jk}, b_{jk}$ sur $U' \cap U''$ tel que
\begin{equation*}
\sigma'_j = \sum_{k=1}^{p''} a_{jk}\sigma''_k \qquad
\sigma''_j = \sum_{k=1}^{p'} b_{jk}\sigma'_k
\end{equation*}

Alors il existe un voisinage $W \subset U' \cup U''$ de $E' \cup E''$ et des sections $\sigma_j \in \mathcal{F}(W)$ pour $1\leq j \leq p=p'+p''$ qui engendre $\mathcal{F}$ sur $W$.
\end{lemme}

\begin{proof}
On pose $\sigma' = (\sigma'_j)$ , $\sigma'' = (\sigma''_j)$ et $A=(a_{jk})$, $B=(b_{jk})$. On a alors
\begin{equation*}
\sigma' = A\sigma'' \qquad
\sigma'' = B\sigma'
\end{equation*}

On va maintenant fusionner ces deux équations en une seule. Pour cela, on pose
\begin{equation*}
\tilde{\sigma}' = \begin{pmatrix}
   \sigma' \\
   \mathbf{0}_{p'}
\end{pmatrix} \qquad
\tilde{\sigma}'' = \begin{pmatrix}
   \mathbf{0}_{p''} \\
   \sigma''
\end{pmatrix}
\end{equation*}
$$ \tilde{A} = \begin{pmatrix}
   I_{p'} & A \\
   -B & I_{p''} - BA 
\end{pmatrix}$$

De l'équation $\sigma'' = BA\sigma''$, on en déduit que
\begin{equation*}
\tilde{A}\sigma'' = \begin{pmatrix}
   I_{p'} & A \\
   -B & I_{p''} - BA 
\end{pmatrix} \begin{pmatrix}
   \mathbf{0}_{p''} \\
   \sigma''
\end{pmatrix} = \begin{pmatrix}
   A\sigma'' \\
   \sigma'' - BA\sigma''
\end{pmatrix} = \begin{pmatrix}
   \sigma' \\
   \mathbf{0}_{p'}
\end{pmatrix}
\end{equation*}

Autrement dit, $\tilde{A}\tilde{\sigma}'' = \tilde{\sigma}'$.

On va aussi poser $P = \begin{pmatrix}
   I_{p'} & A \\
   \mathbf{0} & I_{p''} 
\end{pmatrix}$
et $Q = \begin{pmatrix}
   I_{p'} & \mathbf{0} \\
   B & I_{p''} 
\end{pmatrix}$. Calculons $Q\tilde{A}P^{-1}$, on a
\begin{eqnarray}
\nonumber Q\tilde{A}P^{-1} &=& \begin{pmatrix}
   I_{p'} & \mathbf{0} \\
   B & I_{p''} 
\end{pmatrix}
\begin{pmatrix}
   I_{p'} & A \\
   -B & I_{p''} - BA 
\end{pmatrix}
\begin{pmatrix}
   I_{p'} & -A \\
   \mathbf{0} & I_{p''} 
\end{pmatrix} \\
\nonumber &=& \begin{pmatrix}
   I_{p'} & \mathbf{0} \\
   B & I_{p''} 
\end{pmatrix}\begin{pmatrix}
   I_{p'} & \mathbf{0} \\
   -B & I_{p''} 
\end{pmatrix} = I_p
\end{eqnarray}

On cherche à étendre les applications holomorphe $P$ et $Q$ à un voisinage de $E' \cup E''$ tout en ne perturbant pas trop l'égalité précédente; pour l'instant $P$ et $Q$ ne sont définit que sur $U' \cap U''$.

On approche les fonctions holomorphes $a_{jk}$, $b_{jk}$ sur un voisinage $W_0 \Subset U \cap U'$ de $E'\cap E''$ par des polynômes $\tilde{a}_{jk}$, $\tilde{b}_{jk}$ tel que $$\hat{A}(z) = \tilde{Q}(z)\tilde{A}(z)\tilde{P}^{-1}(z) \in V_0$$ pour tout $z \in W_0$ où $V_0$ est le voisinage qui apparait dans la décomposition de Cartan et l'on note $\tilde{P}$, $\tilde{Q}$ les matrices obtenu en remplaçant les coefficients $a_{jk}$, $b_{jk}$ par les polynômes $\tilde{a}_{jk}$, $\tilde{b}_{jk}$.

D'après la décomposition de Cartan, il existe des applications holomorphes $A' : W' \to GL_p(\mathbb{C})$ et $A'' : W'' \to GL_p(\mathbb{C})$ où $W' \subset U'$ et $W'' \subset U''$ sont des voisinages de $E'$ et $E''$ telles que $$\hat{A} = (A')^{-1}A''$$ sur $W' \cap W''$.

On a $\tilde{Q}\tilde{A}\tilde{P}^{-1} = (A')^{-1}A''$, donc $\tilde{A} = \tilde{Q}^{-1}(A')^{-1}A''\tilde{P}$. Comme $\tilde{A}\sigma'' = \sigma'$, on en déduit que $$A''\tilde{P}\tilde{\sigma}''=A'\tilde{Q}\tilde{\sigma}'$$ sur $W' \cap W''$. Le premier terme de l'égalité étant définit sur $W''$ et le deuxième sur $W'$, on peut définir $\sigma_j \in \mathcal{F}(W'\cup W'')$ par les formules suivantes
\begin{equation*}
\begin{pmatrix}
   \sigma_1 \\ \vdots \\ \sigma_p
\end{pmatrix} = \left\{
\begin{array}{l}
  A'\tilde{Q}\tilde{\sigma}' \quad\text{ sur } W'\\
  A''\tilde{P}\tilde{\sigma}'' \quad\text{ sur } W''
\end{array}
\right.
\end{equation*}

De plus, $\sigma'$ engendre $\mathcal{F}$ sur $W' \subset U'$ et $A'\tilde{Q}$ est inversible donc $\sigma$ engendre $\mathcal{F}$ sur $W'$. De même, $\sigma$ engendre $\mathcal{F}$ sur $W''$; on en déduit que $\sigma$ engendre $\mathcal{F}$ sur $W'\cup W''$.
\end{proof}

\begin{lemme}[Sysygy d'Oka]
Soit $E \Subset \mathbb{C}^n$ un cube fermé.
\begin{enumerate}
\item Soit $\mathcal{F}$ un faisceau cohérent sur un voisinage $V$ de E, il existe un voisinage ouvert $E \subset U \subset V$ tel que $\mathcal{F}$ soit engendré par un nombre fini de section sur $U$. On en déduit alors l'existence de $N \in \mathbb{N}$ et d'une suite exacte $$\mathcal{O}^N_U \to \mathcal{F}_{|U} \to 0$$
\item Soit $\sigma_j \in \mathcal{F}(U)$ un système générateur fini de $\mathcal{F}$ sur $U$ et  soit $\sigma \in \mathcal{F}(U)$. Alors il existe un voisinage ouvert $E \subset U' \subset U$ et des fonctions holomorphes $a_j \in \mathcal{O}(U')$ tel que $$\sigma = \sum_{j=1}^N a_j\sigma_j$$
\end{enumerate}
\end{lemme}

\begin{proof}
Si dim $E = 0$, $E$ est réduit à un point $x$.

D'après la cohérence de $\mathcal{F}$, il existe $N \in \mathbb{N}$ et $\sigma_{j,x} \in \mathcal{F}_x$ tel que le morphisme $\mathcal{O}^N_x \to \mathcal{F}_x$ soit surjectif, donc il existe un voisinage $U$ de $x$ tel que $\mathcal{O}^N_U \to \mathcal{F}_{|U}$ soit surjectif.
Par surjectivité de $\mathcal{O}^N_x \to \mathcal{F}_x$, il existe $a_{j,x} \in \mathcal{O}_x$ tel que $\sigma_x = \sum_{j=1}^N a_{j,x}\sigma_{j,x}$. Quitte à réduire $U$, on a alors $\sigma = \sum_{j=1}^N a_j\sigma_j$ sur $U$.

Supposons maintenant que $\nu = $ dim $E \geq 1$, et supposons le lemme soit vrai pour tout les cubes fermés de dimension $\leq \nu - 1$.

On écrit $E$ sous la forme $E = F \times [0, T]$ où $T > 0$ et $F$ est un cube de dimension $\leq \nu - 1$. D'après l'hypothèse de récurrence, on sait que pour tout $t \in [0, T]$, il existe un voisinage ouvert $U_t$ de $E_t = F \times {t}$ tel que $\mathcal{F}$ soit engendré par un nombre fini de sections sur $U_t$.

Par compacité de $[0, T]$, il existe un nombre fini d'ouvert $U_t$ qui recouvre E, donc il existe $$0 = t_0 < t_1 < ... < t_L = T$$ tel que ${U_{t_j}}$ recouvre $E$. De plus, pour tout $0 \leq j \leq L$ il existe un système générateur fini $(\sigma_{\alpha j})_\alpha$ de $\mathcal{F}$ sur un voisinage de $E_\alpha = F \times [t_{\alpha - 1}, t_\alpha]$.

On a $E_\alpha \cap E_{\alpha+1} = E_{t_\alpha}$ de dimension $\leq \nu - 1$. D'après l'hypothèse de récurrence, il existe des fonctions holomorphes $a_{jk}, b_{jk} \in \mathcal{O}(U_\alpha \cap U_{\alpha +1})$ tel que, sur $U_\alpha \cap U_{\alpha +1}$, on a
\begin{equation*}
\sigma_{\alpha,j} = \sum a_{jk}\sigma_{\alpha+1,j} \qquad
\sigma_{\alpha+1,j} = \sum b_{jk}\sigma_{\alpha,j}
\end{equation*}

On va utiliser le lemme de fusion pour fusionner les systèmes générateurs de $E_1, ..., E_L$. On commence par l'appliquer à $E_1$ et $E_2$; on trouve alors un système générateur fini de $\mathcal{F}$ sur un voisinage de $E_1 \cup E_2$. On l'applique ensuite à $E_1 \cup E_2$ et $E_3$, on trouve ainsi un système générateur fini de $\mathcal{F}$ sur un voisinage de $E_1 \cup E_2 \cup E_3$. Par récurrence, on obtient alors un système générateur fini sur un voisinage de $E=E_1 \cup ... \cup E_L$.

Il nous faut construire des fonctions holomorphes $a_j$ tel que 
$$\sigma = \sum_{j=1}^N a_j\sigma_j$$ sur un voisinage de $E$. D'après l'hypothèse de récurrence, pour tout $t \in [0, T]$, il existe des fonctions holomorphe $a_{t,j}$ sur un voisinage $E_t \subset U'_t \subset U_t$ tel que
$$\sigma = \sum_{j=1}^N a_j\sigma_j$$
sur $U'_t$. Par le même argument de compacité que précédemment, il existe $$0 = t_0 < t_1 < ... < t_L = T$$ tel que ${U_{t_j}}$ recouvre $E$. De plus, il existe des fonctions holomorphes $a_{\alpha,j} \in \mathcal{U_\alpha}$ tel que $$\sigma = \sum_{j=1}^N a_{j, \alpha}\sigma_j$$ sur un voisinage $U_\alpha$ de $E_\alpha$.

On va modifier les fonctions holomorphes $(a_{j, \alpha})_\alpha$ pour pouvoir les recoller sur un voisinage de $E$, on veut de plus qu'après les avoir modifier on garde toujours la même égalité.

Soit $\mathcal{R} = \mathcal{R}(\sigma_j)$ le faisceau des relations, c'est un sous-faisceau localement fini de $\mathcal{O}^N$; donc $\mathcal{R}$ est cohérent sur un voisinage de $E$. D'après la première partie, $\mathcal{R}$ est engendré par un nombre fini de sections $(\tau_h)$ sur un voisinage de $E$. De plus, sur $U_\alpha \cap U_\beta$, on a $$\sum_j a_{\alpha,j}\sigma_j =\sum_j a_{\beta,j}\sigma_j = \sigma$$

On en déduit alors que
$$\sum_j (a_{\alpha,j} - a_{\beta,j})\sigma_j = 0$$ sur $U_\alpha \cap U_\beta$. On pose alors $b_{\alpha, \beta, j} = a_{\alpha,j} - a_{\beta,j}$, on a $(b_{\alpha, \beta, j})_j \in \mathcal{R}(U_\alpha \cap U_\beta)$.

On applique l'hypothèse de récurrence sur $E_\alpha \cap E_\beta$, il existe des fonctions holomorphes $c_{\alpha,\beta,h} \in \mathcal{O}(U_\alpha \cap U_\beta)$ tel que
$$(b_{\alpha,\beta,j})_j = \sum_h c_{\alpha,\beta,h}\tau_h$$

On choisit un cylindre convexe $\Omega$ voisinage de $E$ tel que ${U_\alpha}$ recouvre $\Omega$. Par définition, $\Omega$ est un produit d'ouvert convexe de $\mathbb{C}$. D'après le théorème de représentation de Riemann, chacun de ces ouverts est biholomorphe à un disque; donc $\Omega$ est biholomorphe à un polydisque. D'après le théorème de Dolbeault, on a $H^1(\Omega, \mathcal{O}_\Omega)=0$; on en déduit que $H^1(\{U_\alpha\}, \mathcal{O}_\Omega)=0$.

On pose $c_{\beta, \alpha, h} = - c_{\alpha, \beta, h}$, alors on a $(c_{\alpha,\beta,h})_{\alpha,\beta} \in Z^1(\{U_\alpha\}, \mathcal{O}_\Omega)$. Par l'annulation de la cohomologie, il existe $d_{\alpha, h} \in Z^0(\{U_\alpha\}, \mathcal{O}_\Omega)$ tel que $$c_{\alpha, \beta, h} = d_{\beta, h} - d_{\alpha, h}$$

On en déduit que $$(a_{\alpha,j} - a_{\beta,j})_j = (b_{\alpha,\beta,j})_j = \sum_h c_{\alpha,\beta,h}\tau_h = \sum_h (d_{\beta, h} - d_{\alpha,h})\tau_h$$

En posant $\tau_h = (\tau_{hj})_j$, on en déduit que
$$a_{\alpha,j} + \sum_h d_{\alpha,h}\tau_{h, j} = a_{\beta,j} + \sum_h d_{\beta,h}\tau_{h, j}$$

Cette relation nous permet de recoller ces fonctions holomorphes en $a_j \in \mathcal{O}(\Omega)$. De plus, sur $U_\alpha$, on a
$$\sum_j a_j\sigma_j = \sum_j a_{\alpha,j}\sigma_j + \sum_h d_{\alpha,h}\sum_j \tau_{hj}\sigma_j = \sigma$$

En effet, on a $\tau_h \in \mathcal{R}$ donc $\sum_j \tau_{hj}\sigma_j = 0$. Ce qui termine la preuve du lemme.
\end{proof}

\begin{lemme}[Syzygies d'Oka]
Soit $\mathcal{F}$ un faisceau cohérent sur un voisinage d'un cube fermé $E \Subset \mathbb{C}^n$. Alors pour tout $p \geq 1$, il existe un voisinage ouvert $U$ de $E$, des entiers $N_1,...,N_p \geq 0$ et une suite exacte
$$0 \to \ker \phi_p \to \mathcal{O}^{N_p}_{|U} \to ... \to \mathcal{O}^{N_1}_{|U} \to \mathcal{F}_{|U} \to 0$$

Ces suites exactes sont appelées syzygies d'Oka.
\end{lemme}

\begin{proof}
D'après le lemme précédent, il existe un voisinage ouvert $U_1$ de $E$ , un entier $N_1 \geq 0$ et une suite exacte
$$\mathcal{O}^{N_1}_{|U_1} \overset{\phi_1}{\to} \mathcal{F}_{|U_1} \to 0$$

On peut la compléter pour donner la suite exacte
$$0 \to \ker \phi_1 \to \mathcal{O}^{N_1}_{|U_1} \overset{\phi_1}{\to} \mathcal{F}_{|U_1} \to 0$$

Le faisceau $\mathcal{F}$ étant cohérent, on en déduit que $ker \phi_1$ est un faisceau cohérent. En appliquant le lemme précédent, il existe un voisinage ouvert $U_2$ de $E$, un entier $N_2 \geq 0$ et une suite exacte
$$\mathcal{O}^{N_2}_{|U_2} \overset{\phi_2}{\to} \ker \phi_1 \to 0$$

Quitte à diminuer $U_1$ et $U_2$, on peut supposer que $U_1 = U_2 = U$ voisinage ouvert de $E$. D'autre part, on a Im $\phi_2 = \ker \phi_1$; on en déduit alors une suite exacte
$$\mathcal{O}^{N_2}_{|U} \overset{\phi_2}{\to} \mathcal{O}^{N_1}_{|U} \overset{\phi_1}{\to} \mathcal{F}_{|U} \to 0$$

Que l'on complète pour donner la suite exacte
$$0 \to \ker \phi_2 \to \mathcal{O}^{N_2}_{|U} \overset{\phi_2}{\to} \mathcal{O}^{N_1}_{|U} \overset{\phi_1}{\to} \mathcal{F}_{|U} \to 0$$

Par une récurrence immédiate, on en déduit le lemme.
\end{proof}

Dans la suite on aura besoin d'un théorème d'annulation de la cohomologie pour des degrés suffisamment grand.
\begin{theoreme}
Soit $X \subset \mathbb{R}^n$ un ouvert et $\mathcal{F}$ un faisceau sur $X$. Alors pour tout $q \geq 2^n$, on a $$H^q(X, \mathcal{F})=0$$
\end{theoreme}

\begin{proof}
On considère un pavage non régulier de $X$ par des cubes fermés, c'est à dire, on recouvre $X$ par des cubes fermés $E_\alpha$ qui n'ont pas de point commun en dehors des points du bord. On considère alors un recouvrement $\{U_\alpha\}$ par des cubes ouverts $X$ tel que $E_\alpha \Subset U_\alpha \Subset X$ et on choisit les $U_\alpha$ suffisamment petit tel que si on prend au moins $2^n+1$ ouvert parmi les $U_\alpha$ alors leur intersection est vide.

Si $q \geq 2^n$, alors on a $C^q(\{U_\alpha\}, \mathcal{F})=0$; d'où $H^q(\{U_\alpha\}, \mathcal{F})=0$. De plus, d'après le théorème de Dolbeault, on a $H^q(U_\alpha, \mathcal{F})=0$ pour tout $\alpha$. On en déduit alors d'après le théorème de Leray que
$$H^q(X, \mathcal{F})=H^q(\{U_\alpha\}, \mathcal{F})=0$$
\end{proof}

On commence par démontrer le lemme fondamental d'Oka pour un faisceau cohérent définit sur un voisinage d'un cube fermé.
\begin{lemme}
Soit $\mathcal{F}$ un faisceau cohérent sur un voisinage d'un cube fermé $E \subset \mathbb{C}^n$. Alors, pour $q \geq 1$, on a
$$H^q(int(E), \mathcal{F})=0$$
où $int(E)$ désigne l'intérieur de $E$.
\end{lemme}

\begin{proof}
On a une suite exacte
$$0 \to \ker \phi_1 \to \mathcal{O}^{N_1}_{|U_1} \to \mathcal{F}_{|U_1} \to 0$$
provenant des syzygies d'Oka. On la restreint sur $U=int(E)$ et on considère la suite exacte longue de cohomologie correspondante
$$H^q(U, \mathcal{O}^{N_1}) \to H^q(U, \mathcal{F}) \to  H^{q+1}(U, \ker \phi_1) \to H^{q+1}(U, \mathcal{O}^{N_1})$$

D'après le théorème de Dolbeault, on a
$$H^q(U, \mathcal{O}^{N_1}) =  H^{q+1}(U, \mathcal{O}^{N_1}) = 0$$

On en déduit que $$H^q(U, \mathcal{F}) =  H^{q+1}(U, \ker \phi_1)$$

On arrive ainsi par l'intermédiaire des syzygies d'Oka à augmenter le degré de la cohomologie. De plus, d'après le théorème précédent la cohomologie est nulle pour des degrés $\geq 2^{2n}$. Il ne nous reste plus qu'à utiliser des syzygies d'Oka de longueur plus grande pour conclure.

D'après l'existence de syzygies d'Oka de longueur 2, on a une suite exacte
$$0 \to \ker \phi_2 \to \mathcal{O}^{N_2}_{|U_2} \overset{\phi_2}{\to} \mathcal{O}^{N_1}_{|U_2} \overset{\phi_1}{\to} \mathcal{F}_{|U_2} \to 0$$

On a donc Im $\phi_2 = \ker \phi_1$; en restreignant à $U$ on a alors une suite exacte
$$0 \to \ker \phi_2 \to \mathcal{O}^{N_2}_{|U} \to \ker \phi_1 \to 0$$

Par le même raisonnement que précédemment, on en déduit que
$$H^{q+1}(U, \ker \phi_1) =  H^{q+2}(U, \ker \phi_2)$$

En utilisant l'existence de sysygies d'Oka de longueur arbitraire et une récurrence immédiate, on en déduit que
$$H^q(U, \mathcal{F}) =  H^{q+p}(U, \ker \phi_p)$$

Pour $p = 2^{2n}-q$, on en déduit d'après le théorème précédent l'annulation de la cohomologie.
\end{proof}

\begin{lemme}[Lemme fondamental d'Oka]
Soit $\Omega \subset \mathbb{C}^n$ un cylindre convexe et $\mathcal{F}$ un faisceau cohérent sur $\Omega$. Alors, pour $q \geq 1$, on a
$$H^q(\Omega, \mathcal{F})=0$$
\end{lemme}

\begin{remarque}
Dans le lemme fondamental d'Oka, on ne suppose plus que le faisceau est définit sur un voisinage de $\bar{\Omega}$. La démonstration du lemme fondamental d'Oka consiste à approcher $\Omega$ par l'intérieur par des cubes et à utiliser le lemme précédent d'annulation de la cohomologie.
\end{remarque}

\begin{proof}
D'après le théorème de représentation de Riemann, $\Omega$ est biholomorphe à un cube ouvert. On supposera, sans perte de généralité, que $\Omega$ est un cube ouvert.

Soit $(\Omega_i)_{i \geq 0}$ une suite croissante de cubes ouvert tel que
\begin{equation*}
\Omega_i \Subset \Omega_{i+1}, \qquad
\Omega = \cup_{i \geq 0} \Omega_i
\end{equation*}

On sait d'après le lemme précédent que
$$H^q(\Omega_i, \mathcal{F})=0$$

On va chercher à approcher tout cocycle $f \in Z^q(\Omega, \mathcal{F})$ par des cobords $\delta g_i \in B^q(\Omega_i, \mathcal{F})$.

Soit $\{U_\alpha\}_\alpha$ un recouvrement ouvert de $\Omega$ par des cubes, $\{U_\alpha \cap \Omega_i\}_\alpha$ est alors un recouvrement de $\Omega_i$. D'après le théorème de Dolbeault, on a
\begin{eqnarray}
\nonumber &&H^q(U_\alpha, \mathcal{F}) = 0 \\
\nonumber &&H^q(U_\alpha \cap \Omega_i, \mathcal{F}) = 0
\end{eqnarray}


D'après le théorème de Leray, on en déduit que
\begin{eqnarray}
\nonumber H^q(\Omega, \mathcal{F}) &=& H^q(\{U_\alpha\}, \mathcal{F}), \\
\nonumber H^q(\Omega_i, \mathcal{F}) &=& H^q(\{U_\alpha \cap \Omega_i\}, \mathcal{F})=0
\end{eqnarray}

Soit $f \in Z^q(\{U_\alpha\}, \mathcal{F})$, on a $f_{|\Omega_i} \in Z^q(\{U_\alpha\cap \Omega_i\}, \mathcal{F})$; d'après l'annulation de la cohomologie, il existe $g_i \in C^{q-1}(\{U_\alpha\cap \Omega_i\}, \mathcal{F})$ tel que
$$f_{|\Omega_i} = \delta g_i$$

On se restreint au cas où $q \geq 2$:

On pose $\tilde{g}_1=g_1$; supposons construit par récurrence $\tilde{g}_i \in  C^{q-1}(\{U_\alpha\cap \Omega_i\}, \mathcal{F})$ tel que
\begin{eqnarray}
\nonumber f_{|\Omega_i} &=& \delta \tilde{g}_i \\
\nonumber \tilde{g_i}_{|\Omega_{i-1}} &=& \tilde{g}_{i-1}
\end{eqnarray}

On cherche à construire $\tilde{g}_{i+1}$ qui coïncide avec $\tilde{g}_i$ sur $\Omega_i$. On a
\begin{eqnarray}
\nonumber \delta(\tilde{g}_i - g_{i+1}{|_{\Omega_{i}}})&=& \delta \tilde{g}_i  - \delta g_{i+1}{|_{\Omega_{i}}}\\
\nonumber &=& 0
\end{eqnarray}

On en déduit qu'il existe $h_{i+1} \in C^{q-2}(\{U_\alpha\cap \Omega_i\}, \mathcal{F})$ tel que
$$\tilde{g}_i - g_{i+1}{|_{\Omega_{i}}} = \delta h_{i+1}$$

On étend $h_{i+1} \in C^{q-2}(\{U_\alpha\cap \Omega_i\}, \mathcal{F})$ sur $\Omega$, en posant
$$(\tilde{h}_{i+1})_{\alpha_1, ..., \alpha_{q-1}} = 0 \quad si \quad U_{\alpha_1}\cap ... \cap U_{\alpha_{q-1}} \not \subset \Omega_i$$

On a alors $\tilde{h}_{i+1} \in C^{q-2}(\{U_\alpha\}, \mathcal{F})$. De plus, on pose
$$\tilde{g}_{i+1} = g_{i+1} + \delta \tilde{h}_{i+1}|_{\Omega_{i+1}}$$

On en déduit alors que
\begin{eqnarray}
\nonumber \delta \tilde{g}_{i+1} &=& f_{|\Omega_{i+1}} \\
\nonumber \tilde{g}_{i+1}|_{\Omega_i} &=& \tilde{g}_i
\end{eqnarray}

On définit alors $\tilde{g} \in C^{q-1}(\{U_\alpha\}, \mathcal{F})$ par recollement des $\tilde{g}_i$. On en déduit que
$$\delta \tilde{g}_{|\Omega_i} = \delta \tilde{g}_i = f_{|\Omega_i}$$

On en conclut que $\delta \tilde{g} = f$.
\end{proof}
\end{document}